%&preformat-disser
\RequirePackage[l2tabu,orthodox]{nag} % Раскомментировав, можно в логе получать рекомендации относительно правильного использования пакетов и предупреждения об устаревших и нерекомендуемых пакетах
% Формат А4, 14pt (ГОСТ Р 7.0.11-2011, 5.3.6)
\documentclass[a4paper,14pt,oneside,openany]{memoir}

\input{common/setup}            % общие настройки шаблона
\input{common/packages}         % Пакеты общие для диссертации и автореферата
\synopsisfalse                      % Этот документ --- не автореферат
\input{Dissertation/dispackages}    % Пакеты для диссертации
\input{Dissertation/userpackages}   % Пакеты для специфических пользовательских задач

\input{Dissertation/setup}      % Упрощённые настройки шаблона

\input{common/newnames}         % Новые переменные, для всего проекта

%%% Основные сведения %%%
\newcommand{\thesisAuthorLastName}{\fixme{Чигринский}}
\newcommand{\thesisAuthorOtherNames}{\fixme{Виктор Владимирович}}
\newcommand{\thesisAuthorInitials}{\fixme{В.\,В.}}
\newcommand{\thesisAuthor}             % Диссертация, ФИО автора
{%
    \texorpdfstring{% \texorpdfstring takes two arguments and uses the first for (La)TeX and the second for pdf
        \thesisAuthorLastName~\thesisAuthorOtherNames% так будет отображаться на титульном листе или в тексте, где будет использоваться переменная
    }{%
        \thesisAuthorLastName, \thesisAuthorOtherNames% эта запись для свойств pdf-файла. В таком виде, если pdf будет обработан программами для сбора библиографических сведений, будет правильно представлена фамилия.
    }
}
\newcommand{\thesisAuthorShort}        % Диссертация, ФИО автора инициалами
{\thesisAuthorInitials~\thesisAuthorLastName}
%\newcommand{\thesisUdk}                % Диссертация, УДК
%{\fixme{xxx.xxx}}
\newcommand{\thesisTitle}              % Диссертация, название
{\fixme{Оптимизация решения задач распознавания в области компьютерного зрения с использованием фундаментальных моделей на примере повторной идентификации объектов в контексте мультикамерного трекинга людей}}
\newcommand{\thesisSpecialtyNumber}    % Диссертация, специальность, номер
{\fixme{1.2.1}}
\newcommand{\thesisSpecialtyTitle}     % Диссертация, специальность, название (название взято с сайта ВАК для примера)
{\fixme{Искусственный интеллект и машинное обучение}}
%% \newcommand{\thesisSpecialtyTwoNumber} % Диссертация, вторая специальность, номер
%% {\fixme{XX.XX.XX}}
%% \newcommand{\thesisSpecialtyTwoTitle}  % Диссертация, вторая специальность, название
%% {\fixme{Теория и~методика физического воспитания, спортивной тренировки,
%% оздоровительной и~адаптивной физической культуры}}
\newcommand{\thesisDegree}             % Диссертация, ученая степень
{\fixme{кандидата физико-математических наук}}
\newcommand{\thesisDegreeShort}        % Диссертация, ученая степень, краткая запись
{\fixme{канд. физ.-мат. наук}}
\newcommand{\thesisCity}               % Диссертация, город написания диссертации
{\fixme{Москва}}
\newcommand{\thesisYear}               % Диссертация, год написания диссертации
{\the\year}
\newcommand{\thesisOrganization}       % Диссертация, организация
{\fixme{Федеральный исследовательский центр
<<Информатика и управление Российской академии наук>>}}
\newcommand{\thesisOrganizationShort}  % Диссертация, краткое название организации для доклада
{\fixme{ФИЦ ИУ РАН}}

\newcommand{\thesisInOrganization}     % Диссертация, организация в предложном падеже: Работа выполнена в ...
{\fixme{федеральном исследовательском центре <<Информатика и управление Российской академии наук>>}}

%% \newcommand{\supervisorDead}{}           % Рисовать рамку вокруг фамилии
\newcommand{\supervisorFio}              % Научный руководитель, ФИО
{\fixme{Матвеев Иван Алексеевич}}
\newcommand{\supervisorRegalia}          % Научный руководитель, регалии
{\fixme{доктор технических наук}}
\newcommand{\supervisorFioShort}         % Научный руководитель, ФИО
{\fixme{И.\,А.~Матвеев}}
\newcommand{\supervisorRegaliaShort}     % Научный руководитель, регалии
{\fixme{д.~т.~н.}}

%% \newcommand{\supervisorTwoDead}{}        % Рисовать рамку вокруг фамилии
%% \newcommand{\supervisorTwoFio}           % Второй научный руководитель, ФИО
%% {\fixme{Фамилия Имя Отчество}}
%% \newcommand{\supervisorTwoRegalia}       % Второй научный руководитель, регалии
%% {\fixme{уч. степень, уч. звание}}
%% \newcommand{\supervisorTwoFioShort}      % Второй научный руководитель, ФИО
%% {\fixme{И.\,О.~Фамилия}}
%% \newcommand{\supervisorTwoRegaliaShort}  % Второй научный руководитель, регалии
%% {\fixme{уч.~ст.,~уч.~зв.}}

\newcommand{\opponentOneFio}           % Оппонент 1, ФИО
{\fixme{Фамилия Имя Отчество}}
\newcommand{\opponentOneRegalia}       % Оппонент 1, регалии
{\fixme{доктор физико-математических наук, профессор}}
\newcommand{\opponentOneJobPlace}      % Оппонент 1, место работы
{\fixme{Не очень длинное название для места работы}}
\newcommand{\opponentOneJobPost}       % Оппонент 1, должность
{\fixme{старший научный сотрудник}}

\newcommand{\opponentTwoFio}           % Оппонент 2, ФИО
{\fixme{Фамилия Имя Отчество}}
\newcommand{\opponentTwoRegalia}       % Оппонент 2, регалии
{\fixme{кандидат физико-математических наук}}
\newcommand{\opponentTwoJobPlace}      % Оппонент 2, место работы
{\fixme{Основное место работы c длинным длинным длинным длинным названием}}
\newcommand{\opponentTwoJobPost}       % Оппонент 2, должность
{\fixme{старший научный сотрудник}}

%% \newcommand{\opponentThreeFio}         % Оппонент 3, ФИО
%% {\fixme{Фамилия Имя Отчество}}
%% \newcommand{\opponentThreeRegalia}     % Оппонент 3, регалии
%% {\fixme{кандидат физико-математических наук}}
%% \newcommand{\opponentThreeJobPlace}    % Оппонент 3, место работы
%% {\fixme{Основное место работы c длинным длинным длинным длинным названием}}
%% \newcommand{\opponentThreeJobPost}     % Оппонент 3, должность
%% {\fixme{старший научный сотрудник}}

\newcommand{\leadingOrganizationTitle} % Ведущая организация, дополнительные строки. Удалить, чтобы не отображать в автореферате
{\fixme{Федеральное государственное бюджетное образовательное учреждение высшего
профессионального образования с~длинным длинным длинным длинным названием}}

\newcommand{\defenseDate}              % Защита, дата
{\fixme{DD mmmmmmmm YYYY~г.~в~XX часов}}
\newcommand{\defenseCouncilNumber}     % Защита, номер диссертационного совета
{\fixme{Д\,123.456.78}}
\newcommand{\defenseCouncilTitle}      % Защита, учреждение диссертационного совета
{\fixme{Название учреждения}}
\newcommand{\defenseCouncilAddress}    % Защита, адрес учреждение диссертационного совета
{\fixme{Адрес}}
\newcommand{\defenseCouncilPhone}      % Телефон для справок
{\fixme{+7~(0000)~00-00-00}}

\newcommand{\defenseSecretaryFio}      % Секретарь диссертационного совета, ФИО
{\fixme{Фамилия Имя Отчество}}
\newcommand{\defenseSecretaryRegalia}  % Секретарь диссертационного совета, регалии
{\fixme{д-р~физ.-мат. наук}}            % Для сокращений есть ГОСТы, например: ГОСТ Р 7.0.12-2011 + http://base.garant.ru/179724/#block_30000

\newcommand{\synopsisLibrary}          % Автореферат, название библиотеки
{\fixme{Название библиотеки}}
\newcommand{\synopsisDate}             % Автореферат, дата рассылки
{\fixme{DD mmmmmmmm}\the\year~года}

% To avoid conflict with beamer class use \providecommand
\providecommand{\keywords}%            % Ключевые слова для метаданных PDF диссертации и автореферата
{}
             % Основные сведения
\input{common/fonts}            % Определение шрифтов (частичное)
%%% Шаблон %%%
\DeclareRobustCommand{\fixme}{\textcolor{red}}  % решаем проблему превращения
                                % названия цвета в результате \MakeUppercase,
                                % http://tex.stackexchange.com/a/187930,
                                % \DeclareRobustCommand protects \fixme
                                % from expanding inside \MakeUppercase
\AtBeginDocument{%
    \setlength{\parindent}{1.25em}                   % Абзацный отступ. Должен быть одинаковым по всему тексту и равен пяти знакам (ГОСТ Р 7.0.11-2011, 5.3.7).
}

%%% Таблицы %%%
\DeclareCaptionLabelSeparator{tabsep}{\tablabelsep} % нумерация таблиц
\DeclareCaptionFormat{split}{\splitformatlabel#1\par\splitformattext#3}

\captionsetup[table]{
        format=\tabformat,                % формат подписи (plain|hang)
        font=normal,                      % нормальные размер, цвет, стиль шрифта
        skip=.0pt,                        % отбивка под подписью
        parskip=.0pt,                     % отбивка между параграфами подписи
        position=above,                   % положение подписи
        justification=\tabjust,           % центровка
        indent=\tabindent,                % смещение строк после первой
        labelsep=tabsep,                  % разделитель
        singlelinecheck=\tabsinglecenter, % не выравнивать по центру, если умещается в одну строку
}

%%% Рисунки %%%
\DeclareCaptionLabelSeparator{figsep}{\figlabelsep} % нумерация рисунков

\captionsetup[figure]{
        format=plain,                     % формат подписи (plain|hang)
        font=normal,                      % нормальные размер, цвет, стиль шрифта
        skip=.0pt,                        % отбивка под подписью
        parskip=.0pt,                     % отбивка между параграфами подписи
        position=below,                   % положение подписи
        singlelinecheck=true,             % выравнивание по центру, если умещается в одну строку
        justification=centerlast,         % центровка
        labelsep=figsep,                  % разделитель
}

%%% Подписи подрисунков %%%
\DeclareCaptionSubType{figure}
\renewcommand\thesubfigure{\asbuk{subfigure}} % нумерация подрисунков
\ifsynopsis
\DeclareCaptionFont{norm}{\fontsize{10pt}{11pt}\selectfont}
\newcommand{\subfigureskip}{2.pt}
\else
\DeclareCaptionFont{norm}{\fontsize{14pt}{16pt}\selectfont}
\newcommand{\subfigureskip}{0.pt}
\fi

\captionsetup[subfloat]{
        labelfont=norm,                 % нормальный размер подписей подрисунков
        textfont=norm,                  % нормальный размер подписей подрисунков
        labelsep=space,                 % разделитель
        labelformat=brace,              % одна скобка справа от номера
        justification=centering,        % центровка
        singlelinecheck=true,           % выравнивание по центру, если умещается в одну строку
        skip=\subfigureskip,            % отбивка над подписью
        parskip=.0pt,                   % отбивка между параграфами подписи
        position=below,                 % положение подписи
}

%%% Настройки ссылок на рисунки, таблицы и др. %%%
% команды \cref...format отвечают за форматирование при помощи команды \cref
% команды \labelcref...format отвечают за форматирование при помощи команды \labelcref

\ifpresentation
\else
    \crefdefaultlabelformat{#2#1#3}

    % Уравнение
    \crefformat{equation}{(#2#1#3)} % одиночная ссылка с приставкой
    \labelcrefformat{equation}{(#2#1#3)} % одиночная ссылка без приставки
    \crefrangeformat{equation}{(#3#1#4) \cyrdash~(#5#2#6)} % диапазон ссылок с приставкой
    \labelcrefrangeformat{equation}{(#3#1#4) \cyrdash~(#5#2#6)} % диапазон ссылок без приставки
    \crefmultiformat{equation}{(#2#1#3)}{ и~(#2#1#3)}{, (#2#1#3)}{ и~(#2#1#3)} % перечисление ссылок с приставкой
    \labelcrefmultiformat{equation}{(#2#1#3)}{ и~(#2#1#3)}{, (#2#1#3)}{ и~(#2#1#3)} % перечисление без приставки

    % Подуравнение
    \crefformat{subequation}{(#2#1#3)} % одиночная ссылка с приставкой
    \labelcrefformat{subequation}{(#2#1#3)} % одиночная ссылка без приставки
    \crefrangeformat{subequation}{(#3#1#4) \cyrdash~(#5#2#6)} % диапазон ссылок с приставкой
    \labelcrefrangeformat{subequation}{(#3#1#4) \cyrdash~(#5#2#6)} % диапазон ссылок без приставки
    \crefmultiformat{subequation}{(#2#1#3)}{ и~(#2#1#3)}{, (#2#1#3)}{ и~(#2#1#3)} % перечисление ссылок с приставкой
    \labelcrefmultiformat{subequation}{(#2#1#3)}{ и~(#2#1#3)}{, (#2#1#3)}{ и~(#2#1#3)} % перечисление без приставки

    % Глава
    \crefformat{chapter}{#2#1#3} % одиночная ссылка с приставкой
    \labelcrefformat{chapter}{#2#1#3} % одиночная ссылка без приставки
    \crefrangeformat{chapter}{#3#1#4 \cyrdash~#5#2#6} % диапазон ссылок с приставкой
    \labelcrefrangeformat{chapter}{#3#1#4 \cyrdash~#5#2#6} % диапазон ссылок без приставки
    \crefmultiformat{chapter}{#2#1#3}{ и~#2#1#3}{, #2#1#3}{ и~#2#1#3} % перечисление ссылок с приставкой
    \labelcrefmultiformat{chapter}{#2#1#3}{ и~#2#1#3}{, #2#1#3}{ и~#2#1#3} % перечисление без приставки

    % Параграф
    \crefformat{section}{#2#1#3} % одиночная ссылка с приставкой
    \labelcrefformat{section}{#2#1#3} % одиночная ссылка без приставки
    \crefrangeformat{section}{#3#1#4 \cyrdash~#5#2#6} % диапазон ссылок с приставкой
    \labelcrefrangeformat{section}{#3#1#4 \cyrdash~#5#2#6} % диапазон ссылок без приставки
    \crefmultiformat{section}{#2#1#3}{ и~#2#1#3}{, #2#1#3}{ и~#2#1#3} % перечисление ссылок с приставкой
    \labelcrefmultiformat{section}{#2#1#3}{ и~#2#1#3}{, #2#1#3}{ и~#2#1#3} % перечисление без приставки

    % Приложение
    \crefformat{appendix}{#2#1#3} % одиночная ссылка с приставкой
    \labelcrefformat{appendix}{#2#1#3} % одиночная ссылка без приставки
    \crefrangeformat{appendix}{#3#1#4 \cyrdash~#5#2#6} % диапазон ссылок с приставкой
    \labelcrefrangeformat{appendix}{#3#1#4 \cyrdash~#5#2#6} % диапазон ссылок без приставки
    \crefmultiformat{appendix}{#2#1#3}{ и~#2#1#3}{, #2#1#3}{ и~#2#1#3} % перечисление ссылок с приставкой
    \labelcrefmultiformat{appendix}{#2#1#3}{ и~#2#1#3}{, #2#1#3}{ и~#2#1#3} % перечисление без приставки

    % Рисунок
    \crefformat{figure}{#2#1#3} % одиночная ссылка с приставкой
    \labelcrefformat{figure}{#2#1#3} % одиночная ссылка без приставки
    \crefrangeformat{figure}{#3#1#4 \cyrdash~#5#2#6} % диапазон ссылок с приставкой
    \labelcrefrangeformat{figure}{#3#1#4 \cyrdash~#5#2#6} % диапазон ссылок без приставки
    \crefmultiformat{figure}{#2#1#3}{ и~#2#1#3}{, #2#1#3}{ и~#2#1#3} % перечисление ссылок с приставкой
    \labelcrefmultiformat{figure}{#2#1#3}{ и~#2#1#3}{, #2#1#3}{ и~#2#1#3} % перечисление без приставки

    % Таблица
    \crefformat{table}{#2#1#3} % одиночная ссылка с приставкой
    \labelcrefformat{table}{#2#1#3} % одиночная ссылка без приставки
    \crefrangeformat{table}{#3#1#4 \cyrdash~#5#2#6} % диапазон ссылок с приставкой
    \labelcrefrangeformat{table}{#3#1#4 \cyrdash~#5#2#6} % диапазон ссылок без приставки
    \crefmultiformat{table}{#2#1#3}{ и~#2#1#3}{, #2#1#3}{ и~#2#1#3} % перечисление ссылок с приставкой
    \labelcrefmultiformat{table}{#2#1#3}{ и~#2#1#3}{, #2#1#3}{ и~#2#1#3} % перечисление без приставки

    % Листинг
    \crefformat{lstlisting}{#2#1#3} % одиночная ссылка с приставкой
    \labelcrefformat{lstlisting}{#2#1#3} % одиночная ссылка без приставки
    \crefrangeformat{lstlisting}{#3#1#4 \cyrdash~#5#2#6} % диапазон ссылок с приставкой
    \labelcrefrangeformat{lstlisting}{#3#1#4 \cyrdash~#5#2#6} % диапазон ссылок без приставки
    \crefmultiformat{lstlisting}{#2#1#3}{ и~#2#1#3}{, #2#1#3}{ и~#2#1#3} % перечисление ссылок с приставкой
    \labelcrefmultiformat{lstlisting}{#2#1#3}{ и~#2#1#3}{, #2#1#3}{ и~#2#1#3} % перечисление без приставки

    % Листинг
    \crefformat{ListingEnv}{#2#1#3} % одиночная ссылка с приставкой
    \labelcrefformat{ListingEnv}{#2#1#3} % одиночная ссылка без приставки
    \crefrangeformat{ListingEnv}{#3#1#4 \cyrdash~#5#2#6} % диапазон ссылок с приставкой
    \labelcrefrangeformat{ListingEnv}{#3#1#4 \cyrdash~#5#2#6} % диапазон ссылок без приставки
    \crefmultiformat{ListingEnv}{#2#1#3}{ и~#2#1#3}{, #2#1#3}{ и~#2#1#3} % перечисление ссылок с приставкой
    \labelcrefmultiformat{ListingEnv}{#2#1#3}{ и~#2#1#3}{, #2#1#3}{ и~#2#1#3} % перечисление без приставки
\fi

%%% Настройки гиперссылок %%%
\ifluatex
    \hypersetup{
        unicode,                % Unicode encoded PDF strings
    }
\fi

\hypersetup{
    linktocpage=true,           % ссылки с номера страницы в оглавлении, списке таблиц и списке рисунков
%    linktoc=all,                % both the section and page part are links
%    pdfpagelabels=false,        % set PDF page labels (true|false)
    plainpages=false,           % Forces page anchors to be named by the Arabic form  of the page number, rather than the formatted form
    colorlinks,                 % ссылки отображаются раскрашенным текстом, а не раскрашенным прямоугольником, вокруг текста
    linkcolor={linkcolor},      % цвет ссылок типа ref, eqref и подобных
    citecolor={citecolor},      % цвет ссылок-цитат
    urlcolor={urlcolor},        % цвет гиперссылок
%    hidelinks,                  % Hide links (removing color and border)
    pdftitle={\thesisTitle},    % Заголовок
    pdfauthor={\thesisAuthor},  % Автор
    pdfsubject={\thesisSpecialtyNumber\ \thesisSpecialtyTitle},      % Тема
%    pdfcreator={Создатель},     % Создатель, Приложение
%    pdfproducer={Производитель},% Производитель, Производитель PDF
    pdfkeywords={\keywords},    % Ключевые слова
    pdflang={ru},
}
\ifnumequal{\value{draft}}{1}{% Черновик
    \hypersetup{
        draft,
    }
}{}

%%% Списки %%%
% Используем короткое тире (endash) для ненумерованных списков (ГОСТ 2.105-95, пункт 4.1.7, требует дефиса, но так лучше смотрится)
\renewcommand{\labelitemi}{\normalfont\bfseries{--}}

% Перечисление строчными буквами латинского алфавита (ГОСТ 2.105-95, 4.1.7)
%\renewcommand{\theenumi}{\alph{enumi}}
%\renewcommand{\labelenumi}{\theenumi)}

% Перечисление строчными буквами русского алфавита (ГОСТ 2.105-95, 4.1.7)
\makeatletter
\AddEnumerateCounter{\asbuk}{\russian@alph}{щ}      % Управляем списками/перечислениями через пакет enumitem, а он 'не знает' про asbuk, потому 'учим' его
\makeatother
%\renewcommand{\theenumi}{\asbuk{enumi}} %первый уровень нумерации
%\renewcommand{\labelenumi}{\theenumi)} %первый уровень нумерации
\renewcommand{\theenumii}{\asbuk{enumii}} %второй уровень нумерации
\renewcommand{\labelenumii}{\theenumii)} %второй уровень нумерации
\renewcommand{\theenumiii}{\arabic{enumiii}} %третий уровень нумерации
\renewcommand{\labelenumiii}{\theenumiii)} %третий уровень нумерации

\setlist{nosep,%                                    % Единый стиль для всех списков (пакет enumitem), без дополнительных интервалов.
    labelindent=\parindent,leftmargin=*%            % Каждый пункт, подпункт и перечисление записывают с абзацного отступа (ГОСТ 2.105-95, 4.1.8)
}

%%% Правильная нумерация приложений, рисунков и формул %%%
%% По ГОСТ 2.105, п. 4.3.8 Приложения обозначают заглавными буквами русского алфавита,
%% начиная с А, за исключением букв Ё, З, Й, О, Ч, Ь, Ы, Ъ.
%% Здесь также переделаны все нумерации русскими буквами.
\ifxetexorluatex
    \makeatletter
    \def\russian@Alph#1{\ifcase#1\or
       А\or Б\or В\or Г\or Д\or Е\or Ж\or
       И\or К\or Л\or М\or Н\or
       П\or Р\or С\or Т\or У\or Ф\or Х\or
       Ц\or Ш\or Щ\or Э\or Ю\or Я\else\xpg@ill@value{#1}{russian@Alph}\fi}
    \def\russian@alph#1{\ifcase#1\or
       а\or б\or в\or г\or д\or е\or ж\or
       и\or к\or л\or м\or н\or
       п\or р\or с\or т\or у\or ф\or х\or
       ц\or ш\or щ\or э\or ю\or я\else\xpg@ill@value{#1}{russian@alph}\fi}
    \def\cyr@Alph#1{\ifcase#1\or
        А\or Б\or В\or Г\or Д\or Е\or Ж\or
        И\or К\or Л\or М\or Н\or
        П\or Р\or С\or Т\or У\or Ф\or Х\or
        Ц\or Ш\or Щ\or Э\or Ю\or Я\else\xpg@ill@value{#1}{cyr@Alph}\fi}
    \def\cyr@alph#1{\ifcase#1\or
        а\or б\or в\or г\or д\or е\or ж\or
        и\or к\or л\or м\or н\or
        п\or р\or с\or т\or у\or ф\or х\or
        ц\or ш\or щ\or э\or ю\or я\else\xpg@ill@value{#1}{cyr@alph}\fi}
    \makeatother
\else
    \makeatletter
    \if@uni@ode
      \def\russian@Alph#1{\ifcase#1\or
        А\or Б\or В\or Г\or Д\or Е\or Ж\or
        И\or К\or Л\or М\or Н\or
        П\or Р\or С\or Т\or У\or Ф\or Х\or
        Ц\or Ш\or Щ\or Э\or Ю\or Я\else\@ctrerr\fi}
    \else
      \def\russian@Alph#1{\ifcase#1\or
        \CYRA\or\CYRB\or\CYRV\or\CYRG\or\CYRD\or\CYRE\or\CYRZH\or
        \CYRI\or\CYRK\or\CYRL\or\CYRM\or\CYRN\or
        \CYRP\or\CYRR\or\CYRS\or\CYRT\or\CYRU\or\CYRF\or\CYRH\or
        \CYRC\or\CYRSH\or\CYRSHCH\or\CYREREV\or\CYRYU\or
        \CYRYA\else\@ctrerr\fi}
    \fi
    \if@uni@ode
      \def\russian@alph#1{\ifcase#1\or
        а\or б\or в\or г\or д\or е\or ж\or
        и\or к\or л\or м\or н\or
        п\or р\or с\or т\or у\or ф\or х\or
        ц\or ш\or щ\or э\or ю\or я\else\@ctrerr\fi}
    \else
      \def\russian@alph#1{\ifcase#1\or
        \cyra\or\cyrb\or\cyrv\or\cyrg\or\cyrd\or\cyre\or\cyrzh\or
        \cyri\or\cyrk\or\cyrl\or\cyrm\or\cyrn\or
        \cyrp\or\cyrr\or\cyrs\or\cyrt\or\cyru\or\cyrf\or\cyrh\or
        \cyrc\or\cyrsh\or\cyrshch\or\cyrerev\or\cyryu\or
        \cyrya\else\@ctrerr\fi}
    \fi
    \makeatother
\fi


%%http://www.linux.org.ru/forum/general/6993203#comment-6994589 (используется totcount)
\makeatletter
\def\formtotal#1#2#3#4#5{%
    \newcount\@c
    \@c\totvalue{#1}\relax
    \newcount\@last
    \newcount\@pnul
    \@last\@c\relax
    \divide\@last 10
    \@pnul\@last\relax
    \divide\@pnul 10
    \multiply\@pnul-10
    \advance\@pnul\@last
    \multiply\@last-10
    \advance\@last\@c
    #2%
    \ifnum\@pnul=1#5\else%
    \ifcase\@last#5\or#3\or#4\or#4\or#4\else#5\fi
    \fi
}
\makeatother

\newcommand{\formbytotal}[5]{\total{#1}~\formtotal{#1}{#2}{#3}{#4}{#5}}

%%% Команды рецензирования %%%
\ifboolexpr{ (test {\ifnumequal{\value{draft}}{1}}) or (test {\ifnumequal{\value{showmarkup}}{1}})}{
        \newrobustcmd{\todo}[1]{\textcolor{red}{#1}}
        \newrobustcmd{\note}[2][]{\ifstrempty{#1}{#2}{\textcolor{#1}{#2}}}
        \newenvironment{commentbox}[1][]%
        {\ifstrempty{#1}{}{\color{#1}}}%
        {}
}{
        \newrobustcmd{\todo}[1]{}
        \newrobustcmd{\note}[2][]{}
        \excludecomment{commentbox}
}
           % Стили общие для диссертации и автореферата
%%% Переопределение именований, если иначе не сработает %%%
%\gappto\captionsrussian{
%    \renewcommand{\chaptername}{Глава}
%    \renewcommand{\appendixname}{Приложение} % (ГОСТ Р 7.0.11-2011, 5.7)
%}

%%% Изображения %%%
\graphicspath{{images/}{Dissertation/images/}}         % Пути к изображениям

%%% Интервалы %%%
%% По ГОСТ Р 7.0.11-2011, пункту 5.3.6 требуется полуторный интервал
%% Реализация средствами класса (на основе setspace) ближе к типографской классике.
%% И правит сразу и в таблицах (если со звёздочкой)
\DoubleSpacing*     % Двойной интервал
%\OnehalfSpacing*    % Полуторный интервал
%\setSpacing{1.42}   % Полуторный интервал, подобный Ворду (возможно, стоит включать вместе с предыдущей строкой)

%%% Макет страницы %%%
% Выставляем значения полей (ГОСТ 7.0.11-2011, 5.3.7)
\geometry{a4paper, top=2cm, bottom=2cm, left=2.5cm, right=1cm, nofoot, nomarginpar} %, heightrounded, showframe
\setlength{\topskip}{0pt}   %размер дополнительного верхнего поля
\setlength{\footskip}{12.3pt} % снимет warning, согласно https://tex.stackexchange.com/a/334346

%%% Выравнивание и переносы %%%
%% http://tex.stackexchange.com/questions/241343/what-is-the-meaning-of-fussy-sloppy-emergencystretch-tolerance-hbadness
%% http://www.latex-community.org/forum/viewtopic.php?p=70342#p70342
\tolerance 1414
\hbadness 1414
\emergencystretch 1.5em % В случае проблем регулировать в первую очередь
\hfuzz 0.3pt
\vfuzz \hfuzz
%\raggedbottom
%\sloppy                 % Избавляемся от переполнений
\clubpenalty=10000      % Запрещаем разрыв страницы после первой строки абзаца
\widowpenalty=10000     % Запрещаем разрыв страницы после последней строки абзаца
\brokenpenalty=4991     % Ограничение на разрыв страницы, если строка заканчивается переносом

%%% Блок управления параметрами для выравнивания заголовков в тексте %%%
\newlength{\otstuplen}
\setlength{\otstuplen}{\theotstup\parindent}
\ifnumequal{\value{headingalign}}{0}{% выравнивание заголовков в тексте
    \newcommand{\hdngalign}{\centering}                % по центру
    \newcommand{\hdngaligni}{}% по центру
    \setlength{\otstuplen}{0pt}
}{%
    \newcommand{\hdngalign}{}                 % по левому краю
    \newcommand{\hdngaligni}{\hspace{\otstuplen}}      % по левому краю
} % В обоих случаях вроде бы без переноса, как и надо (ГОСТ Р 7.0.11-2011, 5.3.5)

%%% Оглавление %%%
\renewcommand{\cftchapterdotsep}{\cftdotsep}                % отбивка точками до номера страницы начала главы/раздела

%% Переносить слова в заголовке не допускается (ГОСТ Р 7.0.11-2011, 5.3.5). Заголовки в оглавлении должны точно повторять заголовки в тексте (ГОСТ Р 7.0.11-2011, 5.2.3). Прямого указания на запрет переносов в оглавлении нет, но по той же логике невнесения искажений в смысл, лучше в оглавлении не переносить:
\setrmarg{2.55em plus1fil}                             %To have the (sectional) titles in the ToC, etc., typeset ragged right with no hyphenation
\renewcommand{\cftchapterpagefont}{\normalfont}        % нежирные номера страниц у глав в оглавлении
\renewcommand{\cftchapterleader}{\cftdotfill{\cftchapterdotsep}}% нежирные точки до номеров страниц у глав в оглавлении
%\renewcommand{\cftchapterfont}{}                       % нежирные названия глав в оглавлении

\ifnumgreater{\value{headingdelim}}{0}{%
    \renewcommand\cftchapteraftersnum{.\space}       % добавляет точку с пробелом после номера раздела в оглавлении
}{}
\ifnumgreater{\value{headingdelim}}{1}{%
    \renewcommand\cftsectionaftersnum{.\space}       % добавляет точку с пробелом после номера подраздела в оглавлении
    \renewcommand\cftsubsectionaftersnum{.\space}    % добавляет точку с пробелом после номера подподраздела в оглавлении
    \renewcommand\cftsubsubsectionaftersnum{.\space} % добавляет точку с пробелом после номера подподподраздела в оглавлении
    \AfterEndPreamble{% без этого polyglossia сама всё переопределяет
        \setsecnumformat{\csname the#1\endcsname.\space}
    }
}{%
    \AfterEndPreamble{% без этого polyglossia сама всё переопределяет
        \setsecnumformat{\csname the#1\endcsname\quad}
    }
}

\renewcommand*{\cftappendixname}{\appendixname\space} % Слово Приложение в оглавлении

%%% Колонтитулы %%%
% Порядковый номер страницы печатают на середине верхнего поля страницы (ГОСТ Р 7.0.11-2011, 5.3.8)
\makeevenhead{plain}{}{\rmfamily\thepage}{}
\makeoddhead{plain}{}{\rmfamily\thepage}{}
\makeevenfoot{plain}{}{}{}
\makeoddfoot{plain}{}{}{}
\pagestyle{plain}

%%% добавить Стр. над номерами страниц в оглавлении
%%% http://tex.stackexchange.com/a/306950
\newif\ifendTOC

\newcommand*{\tocheader}{
\ifnumequal{\value{pgnum}}{1}{%
    \ifendTOC\else\hbox to \linewidth%
      {\noindent{}~\hfill{Стр.}}\par%
      \ifnumless{\value{page}}{3}{}{%
        \vspace{0.5\onelineskip}
      }
      \afterpage{\tocheader}
    \fi%
}{}%
}%

%%% Оформление заголовков глав, разделов, подразделов %%%
%% Работа должна быть выполнена ... размером шрифта 12-14 пунктов (ГОСТ Р 7.0.11-2011, 5.3.8). То есть не должно быть надписей шрифтом более 14. Так и поставим.
%% Эти установки будут давать одинаковый результат независимо от выбора базовым шрифтом 12 пт или 14 пт
\newcommand{\basegostsectionfont}{\fontsize{14pt}{16pt}\selectfont\bfseries}

\makechapterstyle{thesisgost}{%
    \chapterstyle{default}
    \setlength{\beforechapskip}{0pt}
    \setlength{\midchapskip}{0pt}
    \setlength{\afterchapskip}{\theintvl\curtextsize}
    \renewcommand*{\chapnamefont}{\basegostsectionfont}
    \renewcommand*{\chapnumfont}{\basegostsectionfont}
    \renewcommand*{\chaptitlefont}{\basegostsectionfont}
    \renewcommand*{\chapterheadstart}{}
    \ifnumgreater{\value{headingdelim}}{0}{%
        \renewcommand*{\afterchapternum}{.\space}   % добавляет точку с пробелом после номера раздела
    }{%
        \renewcommand*{\afterchapternum}{\quad}     % добавляет \quad после номера раздела
    }
    \renewcommand*{\printchapternum}{\hdngaligni\hdngalign\chapnumfont \thechapter}
    \renewcommand*{\printchaptername}{}
    \renewcommand*{\printchapternonum}{\hdngaligni\hdngalign}
}

\makeatletter
\makechapterstyle{thesisgostchapname}{%
    \chapterstyle{thesisgost}
    \renewcommand*{\printchapternum}{\chapnumfont \thechapter}
    \renewcommand*{\printchaptername}{\hdngaligni\hdngalign\chapnamefont \@chapapp} %
}
\makeatother

\chapterstyle{thesisgost}

\setsecheadstyle{\basegostsectionfont\hdngalign}
\setsecindent{\otstuplen}

\setsubsecheadstyle{\basegostsectionfont\hdngalign}
\setsubsecindent{\otstuplen}

\setsubsubsecheadstyle{\basegostsectionfont\hdngalign}
\setsubsubsecindent{\otstuplen}

\sethangfrom{\noindent #1} %все заголовки подразделов центрируются с учетом номера, как block

\ifnumequal{\value{chapstyle}}{1}{%
    \chapterstyle{thesisgostchapname}
    \renewcommand*{\cftchaptername}{\chaptername\space} % будет вписано слово Глава перед каждым номером раздела в оглавлении
}{}%

%%% Интервалы между заголовками
\setbeforesecskip{\theintvl\curtextsize}% Заголовки отделяют от текста сверху и снизу тремя интервалами (ГОСТ Р 7.0.11-2011, 5.3.5).
\setaftersecskip{\theintvl\curtextsize}
\setbeforesubsecskip{\theintvl\curtextsize}
\setaftersubsecskip{\theintvl\curtextsize}
\setbeforesubsubsecskip{\theintvl\curtextsize}
\setaftersubsubsecskip{\theintvl\curtextsize}

%%% Вертикальные интервалы глав (\chapter) в оглавлении как и у заголовков
% раскомментировать следующие 2
% \setlength{\cftbeforechapterskip}{0pt plus 0pt}   % ИЛИ эти 2 строки из учебника
% \renewcommand*{\insertchapterspace}{}
% или эту
% \renewcommand*{\cftbeforechapterskip}{0em}


%%% Блок дополнительного управления размерами заголовков
\ifnumequal{\value{headingsize}}{1}{% Пропорциональные заголовки и базовый шрифт 14 пт
    \renewcommand{\basegostsectionfont}{\large\bfseries}
    \renewcommand*{\chapnamefont}{\Large\bfseries}
    \renewcommand*{\chapnumfont}{\Large\bfseries}
    \renewcommand*{\chaptitlefont}{\Large\bfseries}
}{}

%%% Счётчики %%%

%% Упрощённые настройки шаблона диссертации: нумерация формул, таблиц, рисунков
\ifnumequal{\value{contnumeq}}{1}{%
    \counterwithout{equation}{chapter} % Убираем связанность номера формулы с номером главы/раздела
}{}
\ifnumequal{\value{contnumfig}}{1}{%
    \counterwithout{figure}{chapter}   % Убираем связанность номера рисунка с номером главы/раздела
}{}
\ifnumequal{\value{contnumtab}}{1}{%
    \counterwithout{table}{chapter}    % Убираем связанность номера таблицы с номером главы/раздела
}{}

\AfterEndPreamble{
%% регистрируем счётчики в системе totcounter
    \regtotcounter{totalcount@figure}
    \regtotcounter{totalcount@table}       % Если иным способом поставить в преамбуле то ошибка в числе таблиц
    \regtotcounter{TotPages}               % Если иным способом поставить в преамбуле то ошибка в числе страниц
    \newtotcounter{totalappendix}
    \newtotcounter{totalchapter}
}
  % Стили для диссертации
\input{Dissertation/userstyles} % Стили для специфических пользовательских задач

%%% Библиография. Выбор движка для реализации %%%
% Здесь только проверка установленного ключа. Сама настройка выбора движка
% размещена в common/setup.tex
\ifnumequal{\value{bibliosel}}{0}{%
    \input{biblio/predefined}   % Встроенная реализация с загрузкой файла через движок bibtex8
}{
    \input{biblio/biblatex}     % Реализация пакетом biblatex через движок biber
}

% Вывести информацию о выбранных опциях в лог сборки
\typeout{Selected options:}
\typeout{Draft mode: \arabic{draft}}
\typeout{Font: \arabic{fontfamily}}
\typeout{AltFont: \arabic{usealtfont}}
\typeout{Bibliography backend: \arabic{bibliosel}}
\typeout{Precompile images: \arabic{imgprecompile}}
% Вывести информацию о версиях используемых библиотек в лог сборки
\listfiles

%%% Управление компиляцией отдельных частей диссертации %%%
% Необходимо сначала иметь полностью скомпилированный документ, чтобы все
% промежуточные файлы были в наличии
% Затем, для вывода отдельных частей можно воспользоваться командой \includeonly
% Ниже примеры использования команды:
%
%\includeonly{Dissertation/part2}
%\includeonly{Dissertation/contents,Dissertation/appendix,Dissertation/conclusion}
%
% Если все команды закомментированы, то документ будет выведен в PDF файл полностью

\begin{document}
%%% Переопределение именований типовых разделов
% https://tex.stackexchange.com/a/156050
\gappto\captionsrussian{\input{common/renames}\unskip} % for polyglossia and babel
\input{common/renames}

%%% Структура диссертации (ГОСТ Р 7.0.11-2011, 4)
% Титульный лист (ГОСТ Р 7.0.11-2001, 5.1)
\thispagestyle{empty}
\begin{center}
\thesisOrganization
\end{center}
%
\vspace{0pt plus4fill} %число перед fill = кратность относительно некоторого расстояния fill, кусками которого заполнены пустые места
% \IfFileExists{images/logo.pdf}{
%   \begin{minipage}[b]{0.5\linewidth}
%     \begin{flushleft}
%       \includegraphics[height=3.5cm]{logo}
%     \end{flushleft}
%   \end{minipage}%
%   \begin{minipage}[b]{0.5\linewidth}
%     \begin{flushright}
%       На правах рукописи\\
% %      \textsl {УДК \thesisUdk}
%     \end{flushright}
%   \end{minipage}
% }
{
\begin{flushright}
На правах рукописи

%\textsl {УДК \thesisUdk}
\end{flushright}
}
%
\vspace{0pt plus6fill} %число перед fill = кратность относительно некоторого расстояния fill, кусками которого заполнены пустые места
\begin{center}
{\large \thesisAuthor}
\end{center}
%
\vspace{0pt plus1fill} %число перед fill = кратность относительно некоторого расстояния fill, кусками которого заполнены пустые места
\begin{center}
\textbf {\large %\MakeUppercase
\thesisTitle}

\vspace{0pt plus2fill} %число перед fill = кратность относительно некоторого расстояния fill, кусками которого заполнены пустые места
{%\small
Специальность \thesisSpecialtyNumber\ "---

<<\thesisSpecialtyTitle>>
}

\ifdefined\thesisSpecialtyTwoNumber
{%\small
Специальность \thesisSpecialtyTwoNumber\ "---

<<\thesisSpecialtyTwoTitle>>
}
\fi

\vspace{0pt plus2fill} %число перед fill = кратность относительно некоторого расстояния fill, кусками которого заполнены пустые места
Диссертация на соискание учёной степени

\thesisDegree
\end{center}
%
\vspace{0pt plus4fill} %число перед fill = кратность относительно некоторого расстояния fill, кусками которого заполнены пустые места
\begin{flushright}
\ifdefined\supervisorTwoFio
Научные руководители:

\supervisorRegalia

\ifdefined\supervisorDead
\framebox{\supervisorFio}
\else
\supervisorFio
\fi

\supervisorTwoRegalia

\ifdefined\supervisorTwoDead
\framebox{\supervisorTwoFio}
\else
\supervisorTwoFio
\fi
\else
Научный руководитель:

\supervisorRegalia

\ifdefined\supervisorDead
\framebox{\supervisorFio}
\else
\supervisorFio
\fi
\fi

\end{flushright}
%
\vspace{0pt plus4fill} %число перед fill = кратность относительно некоторого расстояния fill, кусками которого заполнены пустые места
{\centering\thesisCity\ "--- \thesisYear\par}
           % Титульный лист
\include{Dissertation/contents}        % Оглавление
\ifnumequal{\value{contnumfig}}{1}{}{\counterwithout{figure}{chapter}}
\ifnumequal{\value{contnumtab}}{1}{}{\counterwithout{table}{chapter}}
\include{Dissertation/introduction}    % Введение
% \ifnumequal{\value{contnumfig}}{1}{\counterwithout{figure}{chapter}
% }{\counterwithin{figure}{chapter}}
% \ifnumequal{\value{contnumtab}}{1}{\counterwithout{table}{chapter}
% }{\counterwithin{table}{chapter}}
% % \chapter{Оформление различных элементов}\label{ch:ch1}

% \section{Форматирование текста}\label{sec:ch1/sec1}

% Мы можем сделать \textbf{жирный текст} и \textit{курсив}.

% \section{Ссылки}\label{sec:ch1/sec2}

% Сошлёмся на библиографию.
% Одна ссылка: \cite[с.~54]{Sokolov}\cite[с.~36]{Gaidaenko}.
% Две ссылки: \cite{Sokolov,Gaidaenko}.
% Ссылка на собственные работы: \cite{vakbib1, confbib2}.
% Много ссылок: %\cite[с.~54]{Lermontov,Management,Borozda} % такой «фокус»
% %вызывает biblatex warning относительно опции sortcites, потому что неясно, к
% %какому источнику относится уточнение о страницах, а bibtex об этой проблеме
% %даже не предупреждает
% \cite{Lermontov, Management, Borozda, Marketing, Constitution, FamilyCode,
%     Gost.7.0.53, Razumovski, Lagkueva, Pokrovski, Methodology, Berestova,
%     Kriger}%
% \ifnumequal{\value{bibliosel}}{0}{% Примеры для bibtex8
%     \cite{Sirotko, Lukina, Encyclopedia, Nasirova}%
% }{% Примеры для biblatex через движок biber
%     \cite{Sirotko2, Lukina2, Encyclopedia2, Nasirova2}%
% }%
% .
% И~ещё немного ссылок:~\cite{Article,Book,Booklet,Conference,Inbook,Incollection,Manual,Mastersthesis,
%     Misc,Phdthesis,Proceedings,Techreport,Unpublished}
% % Следует обратить внимание, что пробел после запятой внутри \cite{}
% % обрабатывается ожидаемо, а пробел перед запятой, может вызывать проблемы при
% % обработке ссылок.
% \cite{medvedev2006jelektronnye, CEAT:CEAT581, doi:10.1080/01932691.2010.513279,
%     Gosele1999161,Li2007StressAnalysis, Shoji199895, test:eisner-sample,
%     test:eisner-sample-shorted, AB_patent_Pomerantz_1968, iofis_patent1960}%
% \ifnumequal{\value{bibliosel}}{0}{% Примеры для bibtex8
% }{% Примеры для biblatex через движок biber
%     \cite{patent2h, patent3h, patent2}%
% }%
% .

% \ifnumequal{\value{bibliosel}}{0}{% Примеры для bibtex8
% Попытка реализовать несколько ссылок на конкретные страницы
% для \texttt{bibtex} реализации библиографии:
% [\citenum{Sokolov}, с.~54; \citenum{Gaidaenko}, с.~36].
% }{% Примеры для biblatex через движок biber
% Несколько источников (мультицитата):
% % Тут специально написано по-разному тире, для демонстрации, что
% % применение специальных тире в настоящий момент в biblatex приводит к непоказу
% % "с.".
% \cites[vii--x, 5, 7]{Sokolov}[v"--~x, 25, 526]{Gaidaenko}[vii--x, 5, 7]{Techreport},
% работает только в \texttt{biblatex} реализации библиографии.
% }%

% Ссылки на собственные работы:~\cite{vakbib1, confbib1}.

% Сошлёмся на приложения: Приложение~\cref{app:A}, Приложение~\cref{app:B2}.

% Сошлёмся на формулу: формула~\cref{eq:equation1}.

% Сошлёмся на изображение: рисунок~\cref{fig:knuth}.

% Стандартной практикой является добавление к ссылкам префикса, характеризующего тип элемента.
% Это не является строгим требованием, но~позволяет лучше ориентироваться в документах большого размера.
% Например, для ссылок на~рисунки используется префикс \textit{fig},
% для ссылки на~таблицу "--- \textit{tab}.

% В таблице \cref{tab:tab_pref} приложения~\cref{app:B4} приведён список рекомендуемых
% к использованию стандартных префиксов.

% В некоторых ситуациях возникает необходимость отойти от требований ГОСТ по оформлению ссылок на
% литературу.
% В таком случае можно воспользоваться дополнительными опциями пакета \verb+biblatex+.

% Например, в ссылке на книгу~\cite{sobenin_kdv} использование опции \verb+maxnames=4+ позволяет
% вывести имена всех четырёх авторов.
% По ГОСТ имена последних трёх авторов опускаются.

% Кроме того, часто возникают проблемы с транслитерованными инициалами. Некоторые буквы русского
% алфавита по правилам транслитерации записываются двумя буквами латинского алфавита (ю-yu, ё-yo и
% т.д.).
% Такие инициалы \verb+biblatex+ будет сокращать до одной буквы, что неверно.
% Поправить его работу можно использовав опцию \verb+giveninits=false+.
% Пример использования этой опции можно видеть в ссылке~\cite{initials}.

% \section{Формулы}\label{sec:ch1/sec3}

% Благодаря пакету \textit{icomma}, \LaTeX~одинаково хорошо воспринимает
% в~качестве десятичного разделителя и запятую (\(3,1415\)), и точку (\(3.1415\)).

% \subsection{Ненумерованные одиночные формулы}\label{subsec:ch1/sec3/sub1}

% Вот так может выглядеть формула, которую необходимо вставить в~строку
% по~тексту: \(x \approx \sin x\) при \(x \to 0\).

% А вот так выглядит ненумерованная отдельностоящая формула c подстрочными
% и надстрочными индексами:
% \[
%     (x_1+x_2)^2 = x_1^2 + 2 x_1 x_2 + x_2^2
% \]

% Формула с неопределенным интегралом:
% \[
%     \int f(\alpha+x)=\sum\beta
% \]

% При использовании дробей формулы могут получаться очень высокие:
% \[
%     \frac{1}{\sqrt{2}+
%         \displaystyle\frac{1}{\sqrt{2}+
%             \displaystyle\frac{1}{\sqrt{2}+\cdots}}}
% \]

% В формулах можно использовать греческие буквы:
% %Все \original... команды заранее, ради этого примера, определены в Dissertation\userstyles.tex
% \[
%     \alpha\beta\gamma\delta\originalepsilon\epsilon\zeta\eta\theta%
%     \vartheta\iota\kappa\varkappa\lambda\mu\nu\xi\pi\varpi\rho\varrho%
%     \sigma\varsigma\tau\upsilon\originalphi\phi\chi\psi\omega\Gamma\Delta%
%     \Theta\Lambda\Xi\Pi\Sigma\Upsilon\Phi\Psi\Omega
% \]
% \[%https://texfaq.org/FAQ-boldgreek
%     \boldsymbol{\alpha\beta\gamma\delta\originalepsilon\epsilon\zeta\eta%
%         \theta\vartheta\iota\kappa\varkappa\lambda\mu\nu\xi\pi\varpi\rho%
%         \varrho\sigma\varsigma\tau\upsilon\originalphi\phi\chi\psi\omega\Gamma%
%         \Delta\Theta\Lambda\Xi\Pi\Sigma\Upsilon\Phi\Psi\Omega}
% \]

% Для добавления формул можно использовать пары \verb+$+\dots\verb+$+ и \verb+$$+\dots\verb+$$+,
% но~они считаются устаревшими.
% Лучше использовать их функциональные аналоги \verb+\(+\dots\verb+\)+ и \verb+\[+\dots\verb+\]+.

% \subsection{Ненумерованные многострочные формулы}\label{subsec:ch1/sec3/sub2}

% Вот так можно написать две формулы, не нумеруя их, чтобы знаки <<равно>> были
% строго друг под другом:
% \begin{align}
%     f_W & =  \min \left( 1, \max \left( 0, \frac{W_{soil} / W_{max}}{W_{crit}} \right)  \right), \nonumber \\
%     f_T & =  \min \left( 1, \max \left( 0, \frac{T_s / T_{melt}}{T_{crit}} \right)  \right), \nonumber
% \end{align}

% Выровнять систему ещё и по переменной \( x \) можно, используя окружение
% \verb|alignedat| из пакета \verb|amsmath|. Вот так:
% \[
% |x| = \left\{
% \begin{alignedat}{2}
%     &&x, \quad &\text{eсли } x\geqslant 0 \\
%     &-&x, \quad & \text{eсли } x<0
% \end{alignedat}
% \right.
% \]
% Здесь первый амперсанд (в исходном \LaTeX\ описании формулы) означает
% выравнивание по~левому краю, второй "--- по~\( x \), а~третий "--- по~слову
% <<если>>. Команда \verb|\quad| делает большой горизонтальный пробел.

% Ещё вариант:
% \[
%     |x|=
%     \begin{cases}
%         \phantom{-}x, \text{если } x \geqslant 0 \\
%         -x, \text{если } x<0
%     \end{cases}
% \]

% Кроме того, для  нумерованных формул \verb|alignedat| делает вертикальное
% выравнивание номера формулы по центру формулы. Например, выравнивание
% компонент вектора:
% \begin{equation}
%     \label{eq:2p3}
%     \begin{alignedat}{2}
%         {\mathbf{N}}_{o1n}^{(j)} = \,{\sin} \phi\,n\!\left(n+1\right)
%         {\sin}\theta\,
%         \pi_n\!\left({\cos} \theta\right)
%         \frac{
%         z_n^{(j)}\!\left( \rho \right)
%         }{\rho}\,
%         &{\boldsymbol{\hat{\mathrm e}}}_{r}\,+   \\
%         +\,
%         {\sin} \phi\,
%         \tau_n\!\left({\cos} \theta\right)
%         \frac{
%         \left[\rho z_n^{(j)}\!\left( \rho \right)\right]^{\prime}
%         }{\rho}\,
%         &{\boldsymbol{\hat{\mathrm e}}}_{\theta}\,+   \\
%         +\,
%         {\cos} \phi\,
%         \pi_n\!\left({\cos} \theta\right)
%         \frac{
%         \left[\rho z_n^{(j)}\!\left( \rho \right)\right]^{\prime}
%         }{\rho}\,
%         &{\boldsymbol{\hat{\mathrm e}}}_{\phi}\:.
%     \end{alignedat}
% \end{equation}

% Ещё об отступах. Иногда для лучшей <<читаемости>> формул полезно
% немного исправить стандартные интервалы \LaTeX\ с учётом логической
% структуры самой формулы. Например в формуле~\cref{eq:2p3} добавлен
% небольшой отступ \verb+\,+ между основными сомножителями, ниже
% результат применения всех вариантов отступа:
% \begin{align*}
%     \backslash!             & \quad f(x) = x^2\! +3x\! +2         \\
%     \mbox{по-умолчанию}     & \quad f(x) = x^2+3x+2               \\
%     \backslash,             & \quad f(x) = x^2\, +3x\, +2         \\
%     \backslash{:}           & \quad f(x) = x^2\: +3x\: +2         \\
%     \backslash;             & \quad f(x) = x^2\; +3x\; +2         \\
%     \backslash \mbox{space} & \quad f(x) = x^2\ +3x\ +2           \\
%     \backslash \mbox{quad}  & \quad f(x) = x^2\quad +3x\quad +2   \\
%     \backslash \mbox{qquad} & \quad f(x) = x^2\qquad +3x\qquad +2
% \end{align*}

% Можно использовать разные математические алфавиты:
% \begin{align}
%     \mathcal{ABCDEFGHIJKLMNOPQRSTUVWXYZ} \nonumber  \\
%     \mathfrak{ABCDEFGHIJKLMNOPQRSTUVWXYZ} \nonumber \\
%     \mathbb{ABCDEFGHIJKLMNOPQRSTUVWXYZ} \nonumber
% \end{align}

% Посмотрим на систему уравнений на примере аттрактора Лоренца:

% \[
% \left\{
% \begin{array}{rl}
%     \dot x = & \sigma (y-x)  \\
%     \dot y = & x (r - z) - y \\
%     \dot z = & xy - bz
% \end{array}
% \right.
% \]

% А для вёрстки матриц удобно использовать многоточия:
% \[
%     \left(
%         \begin{array}{ccc}
%             a_{11} & \ldots & a_{1n} \\
%             \vdots & \ddots & \vdots \\
%             a_{n1} & \ldots & a_{nn} \\
%         \end{array}
%     \right)
% \]

% \subsection{Нумерованные формулы}\label{subsec:ch1/sec3/sub3}

% А вот так пишется нумерованная формула:
% \begin{equation}
%     \label{eq:equation1}
%     e = \lim_{n \to \infty} \left( 1+\frac{1}{n} \right) ^n
% \end{equation}

% Нумерованных формул может быть несколько:
% \begin{equation}
%     \label{eq:equation2}
%     \lim_{n \to \infty} \sum_{k=1}^n \frac{1}{k^2} = \frac{\pi^2}{6}
% \end{equation}

% Впоследствии на формулы~\cref{eq:equation1, eq:equation2} можно ссылаться.

% Сделать так, чтобы номер формулы стоял напротив средней строки, можно,
% используя окружение \verb|multlined| (пакет \verb|mathtools|) вместо
% \verb|multline| внутри окружения \verb|equation|. Вот так:
% \begin{equation} % \tag{S} % tag - вписывает свой текст
%     \label{eq:equation3}
%     \begin{multlined}
%         1+ 2+3+4+5+6+7+\dots + \\
%         + 50+51+52+53+54+55+56+57 + \dots + \\
%         + 96+97+98+99+100=5050
%     \end{multlined}
% \end{equation}

% Уравнения~\cref{eq:subeq_1,eq:subeq_2} демонстрируют возможности
% окружения \verb|\subequations|.
% \begin{subequations}
%     \label{eq:subeq_1}
%     \begin{gather}
%         y = x^2 + 1 \label{eq:subeq_1-1} \\
%         y = 2 x^2 - x + 1 \label{eq:subeq_1-2}
%     \end{gather}
% \end{subequations}
% Ссылки на отдельные уравнения~\cref{eq:subeq_1-1,eq:subeq_1-2,eq:subeq_2-1}.
% \begin{subequations}
%     \label{eq:subeq_2}
%     \begin{align}
%         y & = x^3 + x^2 + x + 1 \label{eq:subeq_2-1} \\
%         y & = x^2
%     \end{align}
% \end{subequations}

% \subsection{Форматирование чисел и размерностей величин}\label{sec:units}

% Числа форматируются при помощи команды \verb|\num|:
% \num{5,3};
% \num{2,3e8};
% \num{12345,67890};
% \num{2,6 d4};
% \num{1+-2i};
% \num{.3e45};
% \num[exponent-base=2]{5 e64};
% \num[exponent-base=2,exponent-to-prefix]{5 e64};
% \num{1.654 x 2.34 x 3.430}
% \num{1 2 x 3 / 4}.
% Для написания последовательности чисел можно использовать команды \verb|\numlist| и \verb|\numrange|:
% \numlist{10;30;50;70}; \numrange{10}{30}.
% Значения углов можно форматировать при помощи команды \verb|\ang|:
% \ang{2.67};
% \ang{30,3};
% \ang{-1;;};
% \ang{;-2;};
% \ang{;;-3};
% \ang{300;10;1}.

% Обратите внимание, что ГОСТ запрещает использование знака <<->> для обозначения отрицательных чисел
% за исключением формул, таблиц и~рисунков.
% Вместо него следует использовать слово <<минус>>.

% Размерности можно записывать при помощи команд \verb|\si| и \verb|\SI|:
% \si{\farad\squared\lumen\candela};
% \si{\joule\per\mole\per\kelvin};
% \si[per-mode = symbol-or-fraction]{\joule\per\mole\per\kelvin};
% \si{\metre\per\second\squared};
% \SI{0.10(5)}{\neper};
% \SI{1.2-3i e5}{\joule\per\mole\per\kelvin};
% \SIlist{1;2;3;4}{\tesla};
% \SIrange{50}{100}{\volt}.
% Список единиц измерений приведён в таблицах~\cref{tab:unit:base,
%     tab:unit:derived,tab:unit:accepted,tab:unit:physical,tab:unit:other}.
% Приставки единиц приведены в~таблице~\cref{tab:unit:prefix}.

% С дополнительными опциями форматирования можно ознакомиться в~описании пакета \texttt{siunitx};
% изменить или добавить единицы измерений можно в~файле \texttt{siunitx.cfg}.

% \begin{table}
%     \centering
%     \captionsetup{justification=centering} % выравнивание подписи по-центру
%     \caption{Основные величины СИ}\label{tab:unit:base}
%     \begin{tabular}{llc}
%         \toprule
%         Название  & Команда                 & Символ         \\
%         \midrule
%         Ампер     & \verb|\ampere| & \si{\ampere}   \\
%         Кандела   & \verb|\candela| & \si{\candela}  \\
%         Кельвин   & \verb|\kelvin| & \si{\kelvin}   \\
%         Килограмм & \verb|\kilogram| & \si{\kilogram} \\
%         Метр      & \verb|\metre| & \si{\metre}    \\
%         Моль      & \verb|\mole| & \si{\mole}     \\
%         Секунда   & \verb|\second| & \si{\second}   \\
%         \bottomrule
%     \end{tabular}
% \end{table}

% \begin{table}
%     \small
%     \centering
%     \begin{threeparttable}% выравнивание подписи по границам таблицы
%         \caption{Производные единицы СИ}\label{tab:unit:derived}
%         \begin{tabular}{llc|llc}
%             \toprule
%             Название       & Команда                 & Символ              & Название & Команда & Символ \\
%             \midrule
%             Беккерель      & \verb|\becquerel| & \si{\becquerel}     &
%             Ньютон         & \verb|\newton| & \si{\newton}                                      \\
%             Градус Цельсия & \verb|\degreeCelsius| & \si{\degreeCelsius} &
%             Ом             & \verb|\ohm| & \si{\ohm}                                         \\
%             Кулон          & \verb|\coulomb| & \si{\coulomb}       &
%             Паскаль        & \verb|\pascal| & \si{\pascal}                                      \\
%             Фарад          & \verb|\farad| & \si{\farad}         &
%             Радиан         & \verb|\radian| & \si{\radian}                                      \\
%             Грей           & \verb|\gray| & \si{\gray}          &
%             Сименс         & \verb|\siemens| & \si{\siemens}                                     \\
%             Герц           & \verb|\hertz| & \si{\hertz}         &
%             Зиверт         & \verb|\sievert| & \si{\sievert}                                     \\
%             Генри          & \verb|\henry| & \si{\henry}         &
%             Стерадиан      & \verb|\steradian| & \si{\steradian}                                   \\
%             Джоуль         & \verb|\joule| & \si{\joule}         &
%             Тесла          & \verb|\tesla| & \si{\tesla}                                       \\
%             Катал          & \verb|\katal| & \si{\katal}         &
%             Вольт          & \verb|\volt| & \si{\volt}                                        \\
%             Люмен          & \verb|\lumen| & \si{\lumen}         &
%             Ватт           & \verb|\watt| & \si{\watt}                                        \\
%             Люкс           & \verb|\lux| & \si{\lux}           &
%             Вебер          & \verb|\weber| & \si{\weber}                                       \\
%             \bottomrule
%         \end{tabular}
%     \end{threeparttable}
% \end{table}

% \begin{table}
%     \centering
%     \begin{threeparttable}% выравнивание подписи по границам таблицы
%         \caption{Внесистемные единицы}\label{tab:unit:accepted}

%         \begin{tabular}{llc}
%             \toprule
%             Название        & Команда                 & Символ          \\
%             \midrule
%             День            & \verb|\day| & \si{\day}       \\
%             Градус          & \verb|\degree| & \si{\degree}    \\
%             Гектар          & \verb|\hectare| & \si{\hectare}   \\
%             Час             & \verb|\hour| & \si{\hour}      \\
%             Литр            & \verb|\litre| & \si{\litre}     \\
%             Угловая минута  & \verb|\arcminute| & \si{\arcminute} \\
%             Угловая секунда & \verb|\arcsecond| & \si{\arcsecond} \\ %
%             Минута          & \verb|\minute| & \si{\minute}    \\
%             Тонна           & \verb|\tonne| & \si{\tonne}     \\
%             \bottomrule
%         \end{tabular}
%     \end{threeparttable}
% \end{table}

% \begin{table}
%     \centering
%     \captionsetup{justification=centering}
%     \caption{Внесистемные единицы, получаемые из эксперимента}\label{tab:unit:physical}
%     \begin{tabular}{llc}
%         \toprule
%         Название                & Команда                 & Символ                 \\
%         \midrule
%         Астрономическая единица & \verb|\astronomicalunit| & \si{\astronomicalunit} \\
%         Атомная единица массы   & \verb|\atomicmassunit| & \si{\atomicmassunit}   \\
%         Боровский радиус        & \verb|\bohr| & \si{\bohr}             \\
%         Скорость света          & \verb|\clight| & \si{\clight}           \\
%         Дальтон                 & \verb|\dalton| & \si{\dalton}           \\
%         Масса электрона         & \verb|\electronmass| & \si{\electronmass}     \\
%         Электрон Вольт          & \verb|\electronvolt| & \si{\electronvolt}     \\
%         Элементарный заряд      & \verb|\elementarycharge| & \si{\elementarycharge} \\
%         Энергия Хартри          & \verb|\hartree| & \si{\hartree}          \\
%         Постоянная Планка       & \verb|\planckbar| & \si{\planckbar}        \\
%         \bottomrule
%     \end{tabular}
% \end{table}

% \begin{table}
%     \centering
%     \begin{threeparttable}% выравнивание подписи по границам таблицы
%         \caption{Другие внесистемные единицы}\label{tab:unit:other}
%         \begin{tabular}{llc}
%             \toprule
%             Название                  & Команда                 & Символ             \\
%             \midrule
%             Ангстрем                  & \verb|\angstrom| & \si{\angstrom}     \\
%             Бар                       & \verb|\bar| & \si{\bar}          \\
%             Барн                      & \verb|\barn| & \si{\barn}         \\
%             Бел                       & \verb|\bel| & \si{\bel}          \\
%             Децибел                   & \verb|\decibel| & \si{\decibel}      \\
%             Узел                      & \verb|\knot| & \si{\knot}         \\
%             Миллиметр ртутного столба & \verb|\mmHg| & \si{\mmHg}         \\
%             Морская миля              & \verb|\nauticalmile| & \si{\nauticalmile} \\
%             Непер                     & \verb|\neper| & \si{\neper}        \\
%             \bottomrule
%         \end{tabular}
%     \end{threeparttable}
% \end{table}

% \begin{table}
%     \small
%     \centering
%     \begin{threeparttable}% выравнивание подписи по границам таблицы
%         \caption{Приставки СИ}\label{tab:unit:prefix}
%         \begin{tabular}{llcc|llcc}
%             \toprule
%             Приставка & Команда                  & Символ      & Степень &
%             Приставка & Команда                  & Символ      & Степень   \\
%             \midrule
%             Иокто     & \verb|\yocto|  & \si{\yocto} & -24     &
%             Дека      & \verb|\deca|  & \si{\deca}  & 1         \\
%             Зепто     & \verb|\zepto|  & \si{\zepto} & -21     &
%             Гекто     & \verb|\hecto|  & \si{\hecto} & 2         \\
%             Атто      & \verb|\atto|  & \si{\atto}  & -18     &
%             Кило      & \verb|\kilo|  & \si{\kilo}  & 3         \\
%             Фемто     & \verb|\femto|  & \si{\femto} & -15     &
%             Мега      & \verb|\mega|  & \si{\mega}  & 6         \\
%             Пико      & \verb|\pico|  & \si{\pico}  & -12     &
%             Гига      & \verb|\giga|  & \si{\giga}  & 9         \\
%             Нано      & \verb|\nano|  & \si{\nano}  & -9      &
%             Терра     & \verb|\tera|  & \si{\tera}  & 12        \\
%             Микро     & \verb|\micro|  & \si{\micro} & -6      &
%             Пета      & \verb|\peta|  & \si{\peta}  & 15        \\
%             Милли     & \verb|\milli|  & \si{\milli} & -3      &
%             Екса      & \verb|\exa|  & \si{\exa}   & 18        \\
%             Санти     & \verb|\centi|  & \si{\centi} & -2      &
%             Зетта     & \verb|\zetta|  & \si{\zetta} & 21        \\
%             Деци      & \verb|\deci| & \si{\deci}  & -1      &
%             Иотта     & \verb|\yotta| & \si{\yotta} & 24        \\
%             \bottomrule
%         \end{tabular}
%     \end{threeparttable}
% \end{table}

% \subsection{Заголовки с формулами: \texorpdfstring{\(a^2 + b^2 = c^2\)}{%
%         a\texttwosuperior\ + b\texttwosuperior\ = c\texttwosuperior},
%     \texorpdfstring{\(\left\vert\textrm{{Im}}\Sigma\left(
%             \protect\varepsilon\right)\right\vert\approx const\)}{|ImΣ (ε)| ≈ const},
%     \texorpdfstring{\(\sigma_{xx}^{(1)}\)}{σ\_\{xx\}\textasciicircum\{(1)\}}
% }\label{subsec:with_math}

% Пакет \texttt{hyperref} берёт текст для закладок в pdf-файле из~аргументов
% команд типа \verb|\section|, которые могут содержать математические формулы,
% а~также изменения цвета текста или шрифта, которые не отображаются в~закладках.
% Чтобы использование формул в заголовках не вызывало в~логе компиляции появление
% предупреждений типа <<\texttt{Token not allowed in~a~PDF string
%     (Unicode):(hyperref) removing...}>>, следует использовать конструкцию
% \verb|\texorpdfstring{}{}|, где в~первых фигурных скобках указывается
% формула, а~во~вторых "--- запись формулы для закладок.

% \section{Рецензирование текста}\label{sec:markup}

% В шаблоне для диссертации и автореферата заданы команды рецензирования.
% Они видны при компиляции шаблона в режиме черновика или при установке
% соответствующей настройки (\verb+showmarkup+) в~файле \verb+common/setup.tex+.

% Команда \verb+\todo+ отмечает текст красным цветом.
% \todo{Например, так.}

% Команда \verb+\note+ позволяет выбрать цвет текста.
% \note{Чёрный, } \note[red]{красный, } \note[green]{зелёный, }
% \note[blue]{синий.} \note[orange]{Обратите внимание на ширину и расстановку
%     формирующихся пробелов, в~результате приведённой записи (зависит также
%     от~применяемого компилятора).}

% Окружение \verb+commentbox+ также позволяет выбрать цвет.

% \begin{commentbox}[red]
%     Красный текст.

%     Несколько параграфов красного текста.
% \end{commentbox}

% \begin{commentbox}[blue]
%     Синяя формула.

%     \begin{equation}
%         \alpha + \beta = \gamma
%     \end{equation}
% \end{commentbox}

% \verb+commentbox+ позволяет закомментировать участок кода в~режиме чистовика.
% Чтобы убрать кусок кода для всех режимов, можно использовать окружение
% \verb+comment+.

% \begin{comment}
% Этот текст всегда скрыт.
% \end{comment}

% \section{Работа со списком сокращений и~условных обозначений}\label{sec:acronyms}

% С помощью пакета \texttt{nomencl} можно создавать удобный сортированный список
% сокращений и условных обозначений во время написания текста. Вызов
% \verb+\nomenclature+ добавляет нужный символ или сокращение с~описанием
% в~список, который затем печатается вызовом \verb+\printnomenclature+
% в~соответствующем разделе.
% Для того, чтобы эти операции прошли, потребуется дополнительный вызов
% \verb+makeindex -s nomencl.ist -o %.nls %.nlo+ в~командной строке, где вместо
% \verb+%+ следует подставить имя главного файла проекта (\verb+dissertation+
% для этого шаблона).
% Затем потребуется один или два дополнительных вызова компилятора проекта.
% \begin{equation}
%     \omega = c k,
% \end{equation}
% где \( \omega \) "--- частота света, \( c \) "--- скорость света, \( k \) "---
% модуль волнового вектора.
% \nomenclature{\(\omega\)}{частота света\nomrefeq}
% \nomenclature{\(c\)}{скорость света\nomrefpage}
% \nomenclature{\(k\)}{модуль волнового вектора\nomrefeqpage}
% Использование
% \begin{verbatim}
% \nomenclature{\(\omega\)}{частота света\nomrefeq}
% \nomenclature{\(c\)}{скорость света\nomrefpage}
% \nomenclature{\(k\)}{модуль волнового вектора\nomrefeqpage}
% \end{verbatim}
% после уравнения добавит в список условных обозначений три записи.
% Ссылки \verb+\nomrefeq+ на последнее уравнение, \verb+\nomrefpage+ "--- на
% страницу, \verb+\nomrefeqpage+ "--- сразу на~последнее уравнение и~на~страницу,
% можно опускать и~не~использовать.

% Группировкой и сортировкой пунктов в списке можно управлять с~помощью указания
% дополнительных аргументов к команде \verb+nomenclature+.
% Например, при вызове
% \begin{verbatim}
% \nomenclature[03]{\( \hbar \)}{постоянная Планка}
% \nomenclature[01]{\( G \)}{гравитационная постоянная}
% \end{verbatim}
% \( G \) будет стоять в списке выше, чем \( \hbar \).
% Для корректных вертикальных отступов между строками в описании лучше
% не~использовать многострочные формулы в~списке обозначений.

% \nomenclature{%
%     \( \begin{rcases}
%         a_n \\
%         b_n
%     \end{rcases} \)%
% }{коэффициенты разложения Ми в дальнем поле соответствующие электрическим и
%     магнитным мультиполям}
% \nomenclature[a\( e \)]{\( {\boldsymbol{\hat{\mathrm e}}} \)}{единичный вектор}
% \nomenclature{\( E_0 \)}{амплитуда падающего поля}
% \nomenclature{\( j \)}{тип функции Бесселя}
% \nomenclature{\( k \)}{волновой вектор падающей волны}
% \nomenclature{%
%     \( \begin{rcases}
%         a_n \\
%         b_n
%     \end{rcases} \)%
% }{и снова коэффициенты разложения Ми в дальнем поле соответствующие
%     электрическим и магнитным мультиполям. Добавлено много текста, так что
%     описание группы условных обозначений значительно превысило высоту этой
%     группы...}
% \nomenclature{\( L \)}{общее число слоёв}
% \nomenclature{\( l \)}{номер слоя внутри стратифицированной сферы}
% \nomenclature{\( \lambda \)}{длина волны электромагнитного излучения в вакууме}
% \nomenclature{\( n \)}{порядок мультиполя}
% \nomenclature{%
%     \( \begin{rcases}
%         {\mathbf{N}}_{e1n}^{(j)} & {\mathbf{N}}_{o1n}^{(j)} \\
%         {\mathbf{M}_{o1n}^{(j)}} & {\mathbf{M}_{e1n}^{(j)}}
%     \end{rcases} \)%
% }{сферические векторные гармоники}
% \nomenclature{\( \mu \)}{магнитная проницаемость в вакууме}
% \nomenclature{\( r, \theta, \phi \)}{полярные координаты}
% \nomenclature{\( \omega \)}{частота падающей волны}

% С помощью \verb+nomenclature+ можно включать в~список сокращения,
% не~используя их~в~тексте.
% % запись сокращения в список происходит командой \nomenclature,
% % а не употреблением самого сокращения
% \nomenclature{FEM}{finite element method, метод конечных элементов}
% \nomenclature{FIT}{finite integration technique, метод конечных интегралов}
% \nomenclature{FMM}{fast multipole method, быстрый метод многополюсника}
% \nomenclature{FVTD}{finite volume time-domain, метод конечных объёмов
%     во~временной области}
% \nomenclature{MLFMA}{multilevel fast multipole algorithm, многоуровневый
%     быстрый алгоритм многополюсника}
% \nomenclature{BEM}{boundary element method, метод граничных элементов}
% \nomenclature{CST MWS}{Computer Simulation Technology Microwave Studio
%     программа для компьютерного моделирования уравнен Максвелла}
% \nomenclature{DDA}{discrete dipole approximation, приближение дискретиных
%     диполей}
% \nomenclature{FDFD}{finite difference frequency domain, метод конечных
%     разностей в~частотной области}
% \nomenclature{FDTD}{finite difference time domain, метод конечных разностей
%     во~временной области}
% \nomenclature{MoM}{method of moments, метод моментов}
% \nomenclature{MSTM}{multiple sphere T-Matrix, метод Т-матриц для множества
%     сфер}
% \nomenclature{PSTD}{pseudospectral time domain method, псевдоспектральный метод
%     во~временной области}
% \nomenclature{TLM}{transmission line matrix method, метод матриц линий передач}

% \FloatBarrier
           % Глава 1
% \chapter{Методология}\label{ch:ch2}

% \chapter{Длинное название главы, в которой мы смотрим на~примеры того, как будут верстаться изображения и~списки}\label{ch:ch2}

% \section{Одиночное изображение}\label{sec:ch2/sec1}

% \begin{figure}[ht]
%     \centerfloat{
%         \includegraphics[scale=0.27]{latex}
%     }
%     \caption{TeX.}\label{fig:latex}
% \end{figure}

% Для выравнивания изображения по-центру используется команда \verb+\centerfloat+, которая является во
% многом улучшенной версией встроенной команды \verb+\centering+.

% \section{Длинное название параграфа, в котором мы узнаём как сделать две картинки с~общим номером и названием}\label{sec:ch2/sect2}

% А это две картинки под общим номером и названием:
% \begin{figure}[ht]
%     \begin{minipage}[b][][b]{0.49\linewidth}\centering
%         \includegraphics[width=0.5\linewidth]{knuth1} \\ а)
%     \end{minipage}
%     \hfill
%     \begin{minipage}[b][][b]{0.49\linewidth}\centering
%         \includegraphics[width=0.5\linewidth]{knuth2} \\ б)
%     \end{minipage}
%     \caption{Очень длинная подпись к изображению,
%         на котором представлены две фотографии Дональда Кнута}
%     \label{fig:knuth}
% \end{figure}

% Те~же~две картинки под~общим номером и~названием,
% но с автоматизированной нумерацией подрисунков:
% \begin{figure}[ht]
%     \centerfloat{
%         \hfill
%         \subcaptionbox[List-of-Figures entry]{Первый подрисунок\label{fig:knuth_2-1}}{%
%             \includegraphics[width=0.25\linewidth]{knuth1}}
%         \hfill
%         \subcaptionbox{\label{fig:knuth_2-2}}{%
%             \includegraphics[width=0.25\linewidth]{knuth2}}
%         \hfill
%         \subcaptionbox{Третий подрисунок, подпись к которому
%             не~помещается на~одной строке}{%
%             \includegraphics[width=0.3\linewidth]{example-image-c}}
%         \hfill
%     }
%     \legend{Подрисуночный текст, описывающий обозначения, например. Согласно
%         ГОСТ 2.105, пункт 4.3.1, располагается перед наименованием рисунка.}
%     \caption[Этот текст попадает в названия рисунков в списке рисунков]{Очень
%         длинная подпись к второму изображению, на~котором представлены две
%         фотографии Дональда Кнута}\label{fig:knuth_2}
% \end{figure}

% На рисунке~\cref{fig:knuth_2-1} показан Дональд Кнут без головного убора.
% На рисунке~\cref{fig:knuth_2}\subcaptionref*{fig:knuth_2-2}
% показан Дональд Кнут в головном уборе.

% \section{Векторная графика}\label{sec:ch2/vector}

% Возможно вставлять векторные картинки, рассчитываемые \LaTeX\ <<на~лету>>
% с~их~предварительной компиляцией. Надписи в таких рисунках будут выполнены
% тем же~шрифтом, который указан для документа в целом.
% На~рисунке~\cref{fig:tikz_example} на~странице~\pageref{fig:tikz_example}
% представлен пример схемы, рассчитываемой пакетом \verb|tikz| <<на~лету>>.
% Для ускорения компиляции, подобные рисунки могут быть <<кешированы>>, что
% определяется настройками в~\verb|common/setup.tex|.
% Причём имя предкомпилированного
% файла и~папка расположения таких файлов могут быть отдельно заданы,
% что удобно, если не~для подготовки диссертации,
% то~для подготовки научных публикаций.
% \begin{figure}[ht]
%     \centerfloat{
%         \ifdefmacro{\tikzsetnextfilename}{\tikzsetnextfilename{tikz_example_compiled}}{}% присваиваемое предкомпилированному pdf имя файла (не обязательно)
%         \input{Dissertation/images/tikz_scheme.tikz}

%     }
%     \legend{}
%     \caption[Пример \texttt{tikz} схемы]{Пример рисунка, рассчитываемого
%         \texttt{tikz}, который может быть предкомпилирован}\label{fig:tikz_example}
% \end{figure}

% Множество программ имеют либо встроенную возможность экспортировать векторную
% графику кодом \verb|tikz|, либо соответствующий пакет расширения.
% Например, в GeoGebra есть встроенный экспорт,
% для Inkscape есть пакет svg2tikz,
% для Python есть пакет tikzplotlib,
% для R есть пакет tikzdevice.

% \begin{figure}[htbp]
%         \centerfloat{
%                 \ifdefmacro{\tikzsetnextfilename}{\tikzsetnextfilename{pic2}}{}%
%                     \input{Dissertation/images/scheme.tikz}
%                 }
%                 \legend{%
%                     \textbf{1} "--- кружок с загогулиной;
%                     \textbf{2} "--- камертоны;
%                     \textbf{3} "--- кресты;
%                     \textbf{4} "--- волны;
%                     \textbf{5} "--- прямоугольники;
%                     \textbf{5} "--- пронзённый стрелой прямоугольник.%
%                 }
%                 \caption{Составная схема \textit{tikz}}\label{fig:scheme-tikz}
% \end{figure}

% На рисунке~\cref{fig:scheme-tikz} представлена составная схема \textit{tikz}.
% Каждый её элемент нарисован в отдельном файле в единичном масштабе.
% Расстановка элементов на рисунке производится при помощи аргументов \texttt{xshift},
% \texttt{yshift}, \texttt{rotate} и \texttt{scale} окружения \texttt{scope}.

% Пример использования библиотеки \textit{circuitikz} изображён на рисунке~\cref{fig:circuitikz}.

% \begin{figure}[htbp]
%         \centerfloat{
%             \input{Dissertation/images/circuit.tikz}
%         }
%         \caption{Схема \textit{circuitikz}}\label{fig:circuitikz}
% \end{figure}

% Красивые графики также можно добавлять при помощи пакета \textit{pgfplot}~(рисунок~\cref{fig:pgfplot}).
% Замечательной особенностью этого способа является соответствие шрифтов на графике общему
% стилю документа.

% \begin{figure}[htbp]
%         \centerfloat{
%             \input{Dissertation/images/plot_csv.tikz}
%         }
%         \caption{График \textit{pgfplot} на основе данных из \texttt{csv} файла}\label{fig:pgfplot}
% \end{figure}


% \section{Пример вёрстки списков}\label{sec:ch2/sec3}

% \noindent Нумерованный список:
% \begin{enumerate}
%     \item Первый пункт.
%     \item Второй пункт.
%     \item Третий пункт.
% \end{enumerate}

% \noindent Маркированный список:
% \begin{itemize}
%     \item Первый пункт.
%     \item Второй пункт.
%     \item Третий пункт.
% \end{itemize}

% \noindent Вложенные списки:
% \begin{itemize}
%     \item Имеется маркированный список.
%           \begin{enumerate}
%               \item В нём лежит нумерованный список,
%               \item в котором
%                     \begin{itemize}
%                         \item лежит ещё один маркированный список.
%                     \end{itemize}
%           \end{enumerate}
% \end{itemize}

% \noindent Нумерованные вложенные списки:
% \begin{enumerate}
%     \item Первый пункт.
%     \item Второй пункт.
%     \item Вообще, по ГОСТ 2.105 первый уровень нумерации
%           (при необходимости ссылки в тексте документа на одно из перечислений)
%           идёт буквами русского или латинского алфавитов,
%           а второй "--- цифрами со~скобками.
%           Здесь отходим от ГОСТ.
%           \begin{enumerate}
%               \item в нём лежит нумерованный список,
%               \item в котором
%                     \begin{enumerate}
%                         \item ещё один нумерованный список,
%                         \item третий уровень нумерации не нормирован ГОСТ 2.105;
%                         \item обращаем внимание на строчность букв,
%                         \item в этом списке
%                               \begin{itemize}
%                                   \item лежит ещё один маркированный список.
%                               \end{itemize}
%                     \end{enumerate}

%           \end{enumerate}

%     \item Четвёртый пункт.
% \end{enumerate}

% \section{Традиции русского набора}

% Много полезных советов приведено в материале
% <<\href{https://kostyrka.ru/main/ru/typesetting-and-typography-crash-course-by-kostyrka/}{Краткий курс благородного набора}>>
% (автор А.\:В.~Костырка).
% Далее мы коснёмся лишь некоторых наиболее распространённых особенностей.

% \subsection{Пробелы}

% В~русском наборе принято:
% \begin{itemize}
%     \item единицы измерения, знак процента отделять пробелами от~числа:
%           10~кВт, 15~\% (согласно ГОСТ 8.417, раздел 8);
%     \item \(\tg 20\text{\textdegree}\), но: 20~{\textdegree}C
%           (согласно ГОСТ 8.417, раздел 8);
%     \item знак номера, параграфа отделять от~числа: №~5, \S~8;
%     \item стандартные сокращения: т.\:е., и~т.\:д., и~т.\:п.;
%     \item неразрывные пробелы в~предложениях.
% \end{itemize}

% \subsection{Математические знаки и символы}

% Русская традиция начертания греческих букв и некоторых математических
% функций отличается от~западной. Это исправляется серией
% \verb|\renewcommand|.
% \begin{itemize}
%     %Все \original... команды заранее, ради этого примера, определены в Dissertation\userstyles.tex
%     \item[До:] \( \originalepsilon \originalge \originalphi\),
%           \(\originalphi \originalleq \originalepsilon\),
%           \(\originalkappa \in \originalemptyset\),
%           \(\originaltan\),
%           \(\originalcot\),
%           \(\originalcsc\).
%     \item[После:] \( \epsilon \ge \phi\),
%           \(\phi \leq \epsilon\),
%           \(\kappa \in \emptyset\),
%           \(\tan\),
%           \(\cot\),
%           \(\csc\).
% \end{itemize}

% Кроме того, принято набирать греческие буквы вертикальными, что
% решается подключением пакета \verb|upgreek| (см. закомментированный
% блок в~\verb|userpackages.tex|) и~аналогичным переопределением в
% преамбуле (см.~закомментированный блок в~\verb|userstyles.tex|). В
% этом шаблоне такие переопределения уже включены.

% Знаки математических операций принято переносить. Пример переноса
% в~формуле~\eqref{eq:equation3}.

% \subsection{Кавычки}
% В английском языке приняты одинарные и двойные кавычки в~виде ‘...’ и~“...”.
% В~России приняты французские («...») и~немецкие („...“) кавычки (они называются
% «ёлочки» и~«лапки», соответственно). ,,Лапки`` обычно используются внутри
% <<ёлочек>>, например, <<... наш гордый ,,Варяг``...>>.

% Французкие левые и правые кавычки набираются
% как лигатуры \verb|<<| и~\verb|>>|, а~немецкие левые
% и правые кавычки набираются как лигатуры \verb|,,| и~\verb|‘‘| (\verb|``|).

% Вместо лигатур или команд с~активным символом "\ можно использовать команды
% \verb|\glqq| и \verb|\grqq| для набора немецких кавычек и команды \verb|\flqq|
% и~\verb|\frqq| для набора французских кавычек. Они определены в пакете
% \verb|babel|.

% \subsection{Тире}
% %  babel+pdflatex по умолчанию, в polyglossia надо включать опцией (и перекомпилировать с удалением временных файлов)
% Команда \verb|"---| используется для печати тире в тексте. Оно может быть
% несколько короче английского длинного тире (подробности в~документации
% русификации babel). Кроме того, команда задаёт небольшую жёсткую отбивку
% от~слова, стоящего перед тире. При этом, само тире не~отрывается от~слова.
% После тире следует такая же отбивка от текста, как и~перед тире. При наборе
% текста между словом и командой, за которым она следует, должен стоять пробел.

% В составных словах, таких, как <<Закон Менделеева"--~Клапейрона>>, для печати
% тире надо использовать команду \verb|"--~|. Она ставит более короткое,
% по~сравнению с~английским, тире и позволяет делать переносы во втором слове.
% При~наборе текста команда \verb|"--~| не отделяется пробелом от слова,
% за~которым она следует (\verb|Менделеева"--~|). Следующее за командой слово
% может быть  отделено от~неё пробелом или перенесено на другую строку.

% Если прямая речь начинается с~абзаца, то перед началом её печатается тире
% командой \verb|"--*|. Она печатает русское тире и жёсткую отбивку нужной
% величины перед текстом.

% \subsection{Дефисы и переносы слов}
% %  babel+pdflatex по умолчанию, в polyglossia надо включать опцией (и перекомпилировать с удалением временных файлов)
% Для печати дефиса в~составных словах введены две команды. Команда~\verb|"~|
% печатает дефис и~запрещает делать переносы в~самих словах, а~команда \verb|"=|
% печатает дефис, оставляя \TeX ’у право делать переносы в~самих словах.

% В отличие от команды \verb|\-|, команда \verb|"-| задаёт место в~слове, где
% можно делать перенос, не~запрещая переносы и~в~других местах слова.

% Команда \verb|""| задаёт место в~слове, где можно делать перенос, причём дефис
% при~переносе в~этом месте не~ставится.

% Команда \verb|",| вставляет небольшой пробел после инициалов с~правом переноса
% в~фамилии.

% \section{Текст из панграмм и формул}

% Любя, съешь щипцы, "--- вздохнёт мэр, "--- кайф жгуч. Шеф взъярён тчк щипцы
% с~эхом гудбай Жюль. Эй, жлоб! Где туз? Прячь юных съёмщиц в~шкаф. Экс-граф?
% Плюш изъят. Бьём чуждый цен хвощ! Эх, чужак! Общий съём цен шляп (юфть) "---
% вдрызг! Любя, съешь щипцы, "--- вздохнёт мэр, "--- кайф жгуч. Шеф взъярён тчк
% щипцы с~эхом гудбай Жюль. Эй, жлоб! Где туз? Прячь юных съёмщиц в~шкаф.
% Экс-граф? Плюш изъят. Бьём чуждый цен хвощ! Эх, чужак! Общий съём цен шляп
% (юфть) "--- вдрызг! Любя, съешь щипцы, "--- вздохнёт мэр, "--- кайф жгуч. Шеф
% взъярён тчк щипцы с~эхом гудбай Жюль. Эй, жлоб! Где туз? Прячь юных съёмщиц
% в~шкаф. Экс-граф? Плюш изъят. Бьём чуждый цен хвощ! Эх, чужак! Общий съём цен
% шляп (юфть) "--- вдрызг! Любя, съешь щипцы, "--- вздохнёт мэр, "--- кайф жгуч.
% Шеф взъярён тчк щипцы с~эхом гудбай Жюль. Эй, жлоб! Где туз? Прячь юных съёмщиц
% в~шкаф. Экс-граф? Плюш изъят. Бьём чуждый цен хвощ! Эх, чужак! Общий съём цен
% шляп (юфть) "--- вдрызг! Любя, съешь щипцы, "--- вздохнёт мэр, "--- кайф жгуч.
% Шеф взъярён тчк щипцы с~эхом гудбай Жюль. Эй, жлоб! Где туз? Прячь юных съёмщиц
% в~шкаф. Экс-граф? Плюш изъят. Бьём чуждый цен хвощ! Эх, чужак! Общий съём цен
% шляп (юфть) "--- вдрызг! Любя, съешь щипцы, "--- вздохнёт мэр, "--- кайф жгуч.
% Шеф взъярён тчк щипцы с~эхом гудбай Жюль. Эй, жлоб! Где туз? Прячь юных съёмщиц
% в~шкаф. Экс-граф? Плюш изъят. Бьём чуждый цен хвощ! Эх, чужак! Общий съём цен
% шляп (юфть) "--- вдрызг! Любя, съешь щипцы, "--- вздохнёт мэр, "--- кайф жгуч.
% Шеф взъярён тчк щипцы с~эхом гудбай Жюль. Эй, жлоб! Где туз? Прячь юных съёмщиц
% в~шкаф. Экс-граф? Плюш изъят. Бьём чуждый цен хвощ! Эх, чужак! Общий съём цен
% шляп (юфть) "--- вдрызг! Любя, съешь щипцы, "--- вздохнёт мэр, "--- кайф жгуч.
% Шеф взъярён тчк щипцы с~эхом гудбай Жюль. Эй, жлоб! Где туз? Прячь юных съёмщиц
% в~шкаф. Экс-граф? Плюш изъят. Бьём чуждый цен хвощ! Эх, чужак! Общий съём цен
% шляп (юфть) "--- вдрызг! Любя, съешь щипцы, "--- вздохнёт мэр, "--- кайф жгуч.
% Шеф взъярён тчк щипцы с~эхом гудбай Жюль. Эй, жлоб! Где туз? Прячь юных съёмщиц
% в~шкаф. Экс-граф? Плюш изъят. Бьём чуждый цен хвощ! Эх, чужак! Общий съём цен
% шляп (юфть) "--- вдрызг! Любя, съешь щипцы, "--- вздохнёт мэр, "--- кайф жгуч.
% Шеф взъярён тчк щипцы с~эхом гудбай Жюль. Эй, жлоб! Где туз? Прячь юных съёмщиц
% в~шкаф. Экс-граф? Плюш изъят. Бьём чуждый цен хвощ! Эх, чужак! Общий съём цен
% шляп (юфть) "--- вдрызг! Любя, съешь щипцы, "--- вздохнёт мэр, "--- кайф жгуч.
% Шеф взъярён тчк щипцы с~эхом гудбай Жюль. Эй, жлоб! Где туз? Прячь юных съёмщиц
% в~шкаф. Экс-граф? Плюш изъят. Бьём чуждый цен хвощ! Эх, чужак! Общий съём цен
% шляп (юфть) "--- вдрызг!Любя, съешь щипцы, "--- вздохнёт мэр, "--- кайф жгуч.
% Шеф взъярён тчк щипцы с~эхом гудбай Жюль. Эй, жлоб! Где туз? Прячь юных съёмщиц
% в~шкаф. Экс-граф? Плюш изъят. Бьём чуждый цен хвощ! Эх, чужак! Общий съём цен

% Ку кхоро адолэжкэнс волуптариа хаж, вим граэко ыкчпэтында ты. Граэкы жэмпэр
% льюкяльиюч квуй ку, аэквюы продыжщэт хаж нэ. Вим ку магна пырикульа, но квюандо
% пожйдонёюм про. Квуй ат рыквюы ёнэрмйщ. Выро аккузата вим нэ.
% \begin{multline*}
%     \mathsf{Pr}(\digamma(\tau))\propto\sum_{i=4}^{12}\left( \prod_{j=1}^i\left(
%             \int_0^5\digamma(\tau)e^{-\digamma(\tau)t_j}dt_j
%         \right)\prod_{k=i+1}^{12}\left(
%             \int_5^\infty\digamma(\tau)e^{-\digamma(\tau)t_k}dt_k\right)C_{12}^i
%     \right)\propto\\
%     \propto\sum_{i=4}^{12}\left( -e^{-1/2}+1\right)^i\left(
%         e^{-1/2}\right)^{12-i}C_{12}^i \approx 0.7605,\quad
%     \forall\tau\neq\overline{\tau}
% \end{multline*}
% Квуй ыёюз омниюм йн. Экз алёквюам кончюлату квуй, ты альяквюам ёнвидюнт пэр.
% Зыд нэ коммодо пробатуж. Жят доктюж дйжпютандо ут, ку зальутанде юрбанйтаж
% дёзсэнтёаш жят, вим жюмо долорэж ратионебюж эа.

% Ад ентэгры корпора жплэндидэ хаж. Эжт ат факэтэ дычэрунт пэржыкюти. Нэ нам
% доминг пэрчёус. Ку квюо ёужто эррэм зючкёпит. Про хабэо альбюкиюс нэ.
% \[
%     \begin{pmatrix}
%         a_{11} & a_{12} & a_{13} \\
%         a_{21} & a_{22} & a_{23}
%     \end{pmatrix}
% \]

% \[
%     \begin{vmatrix}
%         a_{11} & a_{12} & a_{13} \\
%         a_{21} & a_{22} & a_{23}
%     \end{vmatrix}
% \]

% \[
%     \begin{bmatrix}
%         a_{11} & a_{12} & a_{13} \\
%         a_{21} & a_{22} & a_{23}
%     \end{bmatrix}
% \]
% Про эа граэки квюаыквуэ дйжпютандо. Ыт вэл тебиквюэ дэфянятйоныс, нам жолюм
% квюандо мандамюч эа. Эож пауло лаудым инкедыринт нэ, пэрпэтюа форынчйбюж пэр
% эю. Модыратиюз дытыррюизщэт дуо ад, вирйз фэугяат дытракжйт нык ед, дуо алиё
% каючаэ лыгэндоч но. Эа мольлиз юрбанйтаж зигнёфэрумквюы эжт.

% Про мандамюч кончэтытюр ед. Трётанё прёнкипыз зигнёфэрумквюы вяш ан. Ат хёз
% эквюедым щуавятатэ. Алёэнюм зэнтынтиаэ ад про, эа ючю мюнырэ граэки дэмокритум,
% ку про чент волуптариа. Ыльит дыкоры аляквюид еюж ыт. Ку рыбюм мюндй ютенам
% дуо.
% \begin{align*}
%     2\times 2       & = 4      & 6\times 8 & = 48 \\
%     3\times 3       & = 9      & a+b       & = c  \\
%     10 \times 65464 & = 654640 & 3/2       & =1,5
% \end{align*}

% \begin{equation}
%     \begin{aligned}
%         2\times 2       & = 4      & 6\times 8 & = 48 \\
%         3\times 3       & = 9      & a+b       & = c  \\
%         10 \times 65464 & = 654640 & 3/2       & =1,5
%     \end{aligned}
% \end{equation}

% Пэр йн тальэ пожтэа, мыа ед попюльо дэбетиз жкрибэнтур. Йн квуй аппэтырэ
% мэнандря, зыд аляквюид хабымуч корпора йн. Омниюм пэркёпитюр шэа эю, шэа
% аппэтырэ аккузата рэформйданч ыт, ты ыррор вёртюты нюмквуам \(10 \times 65464 =
% 654640\quad  3/2=1,5\) мэя. Ипзум эуежмод \(a+b = c\) мальюизчыт ад дуо. Ад
% фэюгаят пытынтёюм адвыржаряюм вяш. Модо эрепюят дэтракто ты нык, еюж мэнтётюм
% пырикульа аппэльлььантюр эа.

% Мэль ты дэлььынётё такематыш. Зэнтынтиаэ конклььюжионэмквуэ ан мэя. Вёжи лебыр
% квюаыквуэ квуй нэ, дуо зймюл дэлььиката ку. Ыам ку алиё путынт.

% %Большая фигурная скобка только справа
% \[\left. %ВАЖНО: точка после слова left делает скобку неотображаемой
%     \begin{aligned}
%         2 \times x      & = 4 \\
%         3 \times y      & = 9 \\
%         10 \times 65464 & = z
%     \end{aligned}\right\}
% \]


% Конвынёры витюпырата но нам, тебиквюэ мэнтётюм позтюлант ед про. Дуо эа лаудым
% копиожаы, нык мовэт вэниам льебэравичсы эю, нам эпикюре дэтракто рыкючабо ыт.
% Вэрйтюж аккюжамюз ты шэа, дэбетиз форынчйбюж жкряпшэрит ыт прё. Ан еюж тымпор
% рыфэррэнтур, ючю дольор котёдиэквюэ йн. Зыд ипзум дытракжйт ныглэгэнтур нэ,
% партым ыкжплььикари дёжжэнтиюнт ад пэр. Мэль ты кытэрож молыжтйаы, нам но ыррор
% жкрипта аппарэат.

% \[ \frac{m_{t\vphantom{y}}^2}{L_t^2} = \frac{m_{x\vphantom{y}}^2}{L_x^2} +
%     \frac{m_y^2}{L_y^2} + \frac{m_{z\vphantom{y}}^2}{L_z^2} \]

% Вэре льаборэж тебиквюэ хаж ут. Ан пауло торквюатоз хаж, нэ пробо фэугяат
% такематыш шэа. Мэльёуз пэртинакёа юлламкорпэр прё ад, но мыа рыквюы конкыптам.
% Хёз квюот пэртинакёа эи, ельлюд трактатоз пэр ад. Зыд ед анёмал льаборэж
% номинави, жят ад конгуы льабятюр. Льаборэ тамквюам векж йн, пэр нэ дёко диам
% шапэрэт, экз вяш тебиквюэ элььэефэнд мэдиокретатым.

% Нэ про натюм фюйзчыт квюальизквюэ, аэквюы жкаывола мэль ку. Ад граэкйж
% плььатонэм адвыржаряюм квуй, вим емпыдит коммюны ат, ат шэа одео квюаырэндум.
% Вёртюты ажжынтиор эффикеэнди эож нэ, доминг лаборамюз эи ыам. Чэнзэрет
% мныжаркхюм экз эож, ыльит тамквюам факильизиж нык эи. Квуй ан элыктрам
% тинкидюнт ентырпрытаряш. Йн янвыняры трактатоз зэнтынтиаэ зыд. Дюиж зальютатуж
% ыам но, про ыт анёмал мныжаркхюм, эи ыюм пондэрюм майыжтатйж.

% \FloatBarrier
           % Глава 2
% % \chapter{Вёрстка таблиц}\label{ch:ch3}

% \section{Таблица обыкновенная}\label{sec:ch3/sect1}

% Так размещается таблица:

% \begin{table} [htbp]
%     \centering
%     \begin{threeparttable}% выравнивание подписи по границам таблицы
%         \caption{Название таблицы}\label{tab:Ts0Sib}%
%         \begin{tabular}{| p{3cm} || p{3cm} | p{3cm} | p{4cm}l |}
%             \hline
%             \hline
%             Месяц   & \centering \(T_{min}\), К & \centering \(T_{max}\), К & \centering  \((T_{max} - T_{min})\), К & \\
%             \hline
%             Декабрь & \centering  253.575       & \centering  257.778       & \centering      4.203                  & \\
%             Январь  & \centering  262.431       & \centering  263.214       & \centering      0.783                  & \\
%             Февраль & \centering  261.184       & \centering  260.381       & \centering     \(-\)0.803              & \\
%             \hline
%             \hline
%         \end{tabular}
%     \end{threeparttable}
% \end{table}

% \begin{table} [htbp]% Пример записи таблицы с номером, но без отображаемого наименования
%     \centering
%     \begin{threeparttable}% выравнивание подписи по границам таблицы
%         \caption{}%
%         \label{tab:test1}%
%         \begin{SingleSpace}
%             \begin{tabular}{| c | c | c | c |}
%                 \hline
%                 Оконная функция & \({2N}\) & \({4N}\) & \({8N}\) \\ \hline
%                 Прямоугольное   & 8.72     & 8.77     & 8.77     \\ \hline
%                 Ханна           & 7.96     & 7.93     & 7.93     \\ \hline
%                 Хэмминга        & 8.72     & 8.77     & 8.77     \\ \hline
%                 Блэкмана        & 8.72     & 8.77     & 8.77     \\ \hline
%             \end{tabular}%
%         \end{SingleSpace}
%     \end{threeparttable}
% \end{table}

% Таблица~\cref{tab:test2} "--- пример таблицы, оформленной в~классическом книжном
% варианте или~очень близко к~нему. \mbox{ГОСТу} по~сути не~противоречит. Можно
% ещё~улучшить представление, с~помощью пакета \verb|siunitx| или~подобного.

% \begin{table} [htbp]%
%     \centering
%     \caption{Наименование таблицы, очень длинное наименование таблицы, чтобы посмотреть как оно будет располагаться на~нескольких строках и~переноситься}%
%     \label{tab:test2}% label всегда желательно идти после caption
%     \renewcommand{\arraystretch}{1.5}%% Увеличение расстояния между рядами, для улучшения восприятия.
%     \begin{SingleSpace}
%         \begin{tabular}{@{}@{\extracolsep{20pt}}llll@{}} %Вертикальные полосы не используются принципиально, как и лишние горизонтальные (допускается по ГОСТ 2.105 пункт 4.4.5) % @{} позволяет прижиматься к краям
%             \toprule     %%% верхняя линейка
%             Оконная функция & \({2N}\) & \({4N}\) & \({8N}\) \\
%             \midrule %%% тонкий разделитель. Отделяет названия столбцов. Обязателен по ГОСТ 2.105 пункт 4.4.5
%             Прямоугольное   & 8.72     & 8.77     & 8.77     \\
%             Ханна           & 7.96     & 7.93     & 7.93     \\
%             Хэмминга        & 8.72     & 8.77     & 8.77     \\
%             Блэкмана        & 8.72     & 8.77     & 8.77     \\
%             \bottomrule %%% нижняя линейка
%         \end{tabular}%
%     \end{SingleSpace}
% \end{table}

% \section{Таблица с многострочными ячейками и примечанием}

% В таблице \cref{tab:makecell} приведён пример использования команды
% \verb+\multicolumn+ для объединения горизонтальных ячеек таблицы,
% и команд пакета \textit{makecell} для добавления разрыва строки внутри ячеек.
% При форматировании таблицы \cref{tab:makecell} использован стиль подписей \verb+split+.
% Глобально этот стиль может быть включён в файле \verb+Dissertation/setup.tex+ для диссертации и в
% файле \verb+Synopsis/setup.tex+ для автореферата.
% Однако такое оформление не~соответствует ГОСТ.

% \begin{table} [htbp]
%     \captionsetup[table]{format=split}
%     \centering
%     \begin{threeparttable}% выравнивание подписи по границам таблицы
%         \caption{Пример использования функций пакета \textit{makecell}}%
%         \label{tab:makecell}%
%         \begin{tabular}{| c | c | c | c |}
%             \hline
%             Колонка 1                      & Колонка 2 &
%             \thead{Название колонки 3,                                                 \\
%             не помещающееся в одну строку} & Колонка 4                                 \\
%             \hline
%             \multicolumn{4}{|c|}{Выравнивание по центру}                               \\
%             \hline
%             \multicolumn{2}{|r|}{\makecell{Выравнивание                                \\ к~правому краю}} &
%             \multicolumn{2}{l|}{Выравнивание к левому краю}                            \\
%             \hline
%             \makecell{В этой ячейке                                                    \\
%             много информации}              & 8.72      & 8.55                   & 8.44 \\
%             \cline{3-4}
%             А в этой мало                  & 8.22      & \multicolumn{2}{c|}{5}        \\
%             \hline
%         \end{tabular}%
%     \end{threeparttable}
% \end{table}

% Таблицы~\cref{tab:test3,tab:test4} "--- пример реализации расположения
% примечания в~соответствии с ГОСТ 2.105. Каждый вариант со своими достоинствами
% и~недостатками. Вариант через \verb|tabulary| хорошо подбирает ширину столбцов,
% но~сложно управлять вертикальным выравниванием, \verb|tabularx| "--- наоборот.
% \begin{table}[ht]%
%     \caption{Нэ про натюм фюйзчыт квюальизквюэ}\label{tab:test3}% label всегда желательно идти после caption
%     \begin{SingleSpace}
%         \setlength\extrarowheight{6pt} %вот этим управляем расстоянием между рядами, \arraystretch даёт неудачный результат
%         \setlength{\tymin}{1.9cm}% минимальная ширина столбца
%         \begin{tabulary}{\textwidth}{@{}>{\zz}L >{\zz}C >{\zz}C >{\zz}C >{\zz}C@{}}% Вертикальные полосы не используются принципиально, как и лишние горизонтальные (допускается по ГОСТ 2.105 пункт 4.4.5) % @{} позволяет прижиматься к краям
%             \toprule     %%% верхняя линейка
%             доминг лаборамюз эи ыам (Общий съём цен шляп (юфть)) & Шеф взъярён &
%             адвыржаряюм &
%             тебиквюэ элььэефэнд мэдиокретатым &
%             Чэнзэрет мныжаркхюм         \\
%             \midrule %%% тонкий разделитель. Отделяет названия столбцов. Обязателен по ГОСТ 2.105 пункт 4.4.5
%             Эй, жлоб! Где туз? Прячь юных съёмщиц в~шкаф Плюш изъят. Бьём чуждый цен хвощ! &
%             \({\approx}\) &
%             \({\approx}\) &
%             \({\approx}\) &
%             \( + \) \\
%             Эх, чужак! Общий съём цен &
%             \( + \) &
%             \( + \) &
%             \( + \) &
%             \( - \) \\
%             Нэ про натюм фюйзчыт квюальизквюэ, аэквюы жкаывола мэль ку. Ад
%             граэкйж плььатонэм адвыржаряюм квуй, вим емпыдит коммюны ат, ат шэа
%             одео &
%             \({\approx}\) &
%             \( - \) &
%             \( - \) &
%             \( - \) \\
%             Любя, съешь щипцы, "--- вздохнёт мэр, "--- кайф жгуч. &
%             \( - \) &
%             \( + \) &
%             \( + \) &
%             \({\approx}\) \\
%             Нэ про натюм фюйзчыт квюальизквюэ, аэквюы жкаывола мэль ку. Ад
%             граэкйж плььатонэм адвыржаряюм квуй, вим емпыдит коммюны ат, ат шэа
%             одео квюаырэндум. Вёртюты ажжынтиор эффикеэнди эож нэ. &
%             \( + \) &
%             \( - \) &
%             \({\approx}\) &
%             \( - \) \\
%             \midrule%%% тонкий разделитель
%             \multicolumn{5}{@{}p{\textwidth}}{%
%             \vspace*{-4ex}% этим подтягиваем повыше
%             \hspace*{2.5em}% абзацный отступ - требование ГОСТ 2.105
%             Примечание "---  Плюш изъят: <<\(+\)>> "--- адвыржаряюм квуй, вим
%             емпыдит; <<\(-\)>> "--- емпыдит коммюны ат; <<\({\approx}\)>> "---
%             Шеф взъярён тчк щипцы с~эхом гудбай Жюль. Эй, жлоб! Где туз?
%             Прячь юных съёмщиц в~шкаф. Экс-граф?
%             }
%             \\
%             \bottomrule %%% нижняя линейка
%         \end{tabulary}%
%     \end{SingleSpace}
% \end{table}

% Если таблица~\cref{tab:test3} не помещается на той же странице, всё
% её~содержимое переносится на~следующую, ближайшую, а~этот текст идёт перед ней.
% \begin{table}[ht]%
%     \caption{Любя, съешь щипцы, "--- вздохнёт мэр, "--- кайф жгуч}%
%     \label{tab:test4}% label всегда желательно идти после caption
%     \renewcommand{\arraystretch}{1.6}%% Увеличение расстояния между рядами, для улучшения восприятия.
%     \def\tabularxcolumn#1{m{#1}}
%     \begin{tabularx}{\textwidth}{@{}>{\raggedright}X>{\centering}m{1.9cm} >{\centering}m{1.9cm} >{\centering}m{1.9cm} >{\centering\arraybackslash}m{1.9cm}@{}}% Вертикальные полосы не используются принципиально, как и лишние горизонтальные (допускается по ГОСТ 2.105 пункт 4.4.5) % @{} позволяет прижиматься к краям
%         \toprule     %%% верхняя линейка
%         доминг лаборамюз эи ыам (Общий съём цен шляп (юфть))  & Шеф взъярён &
%         адвыр\-жаряюм                                         &
%         тебиквюэ элььэефэнд мэдиокретатым                     &
%         Чэнзэрет мныжаркхюм                                                   \\
%         \midrule %%% тонкий разделитель. Отделяет названия столбцов. Обязателен по ГОСТ 2.105 пункт 4.4.5
%         Эй, жлоб! Где туз? Прячь юных съёмщиц в~шкаф Плюш изъят.
%         Бьём чуждый цен хвощ!                                 &
%         \({\approx}\)                                         &
%         \({\approx}\)                                         &
%         \({\approx}\)                                         &
%         \( + \)                                                               \\
%         Эх, чужак! Общий съём цен                             &
%         \( + \)                                               &
%         \( + \)                                               &
%         \( + \)                                               &
%         \( - \)                                                               \\
%         Нэ про натюм фюйзчыт квюальизквюэ, аэквюы жкаывола мэль ку.
%         Ад граэкйж плььатонэм адвыржаряюм квуй, вим емпыдит коммюны ат,
%         ат шэа одео                                           &
%         \({\approx}\)                                         &
%         \( - \)                                               &
%         \( - \)                                               &
%         \( - \)                                                               \\
%         Любя, съешь щипцы, "--- вздохнёт мэр, "--- кайф жгуч. &
%         \( - \)                                               &
%         \( + \)                                               &
%         \( + \)                                               &
%         \({\approx}\)                                                         \\
%         Нэ про натюм фюйзчыт квюальизквюэ, аэквюы жкаывола мэль ку. Ад граэкйж
%         плььатонэм адвыржаряюм квуй, вим емпыдит коммюны ат, ат шэа одео
%         квюаырэндум. Вёртюты ажжынтиор эффикеэнди эож нэ.     &
%         \( + \)                                               &
%         \( - \)                                               &
%         \({\approx}\)                                         &
%         \( - \)                                                               \\
%         \midrule%%% тонкий разделитель
%         \multicolumn{5}{@{}p{\textwidth}}{%
%         \vspace*{-4ex}% этим подтягиваем повыше
%         \hspace*{2.5em}% абзацный отступ - требование ГОСТ 2.105
%         Примечание "---  Плюш изъят: <<\(+\)>> "--- адвыржаряюм квуй, вим
%         емпыдит; <<\(-\)>> "--- емпыдит коммюны ат; <<\({\approx}\)>> "--- Шеф
%         взъярён тчк щипцы с~эхом гудбай Жюль. Эй, жлоб! Где туз? Прячь юных
%         съёмщиц в~шкаф. Экс-граф?
%         }
%         \\
%         \bottomrule %%% нижняя линейка
%     \end{tabularx}%
% \end{table}

% \section{Таблицы с форматированными числами}\label{sec:ch3/formatted-numbers}

% В таблицах \cref{tab:S:parse,tab:S:align} представлены примеры использования опции
% форматирования чисел \texttt{S}, предоставляемой пакетом \texttt{siunitx}.

% \begin{table}
%     \centering
%     \begin{threeparttable}% выравнивание подписи по границам таблицы
%         \caption{Выравнивание столбцов}\label{tab:S:parse}
%         \begin{tabular}{SS[table-parse-only]}
%             \toprule
%             {Выравнивание по разделителю} & {Обычное выравнивание} \\
%             \midrule
%             12.345                        & 12.345                 \\
%             6,78                          & 6,78                   \\
%             -88.8(9)                      & -88.8(9)               \\
%             4.5e3                         & 4.5e3                  \\
%             \bottomrule
%         \end{tabular}
%     \end{threeparttable}
% \end{table}

% \begin{table}
%     \centering
%     \begin{threeparttable}% выравнивание подписи по границам таблицы
%         \caption{Выравнивание с использованием опции \texttt{S}}\label{tab:S:align}
%         \sisetup{
%             table-figures-integer = 2,
%             table-figures-decimal = 4
%         }
%         \begin{tabular}
%             {SS[table-number-alignment = center]S[table-number-alignment = left]S[table-number-alignment = right]}
%             \toprule
%             {Колонка 1} & {Колонка 2} & {Колонка 3} & {Колонка 4} \\
%             \midrule
%             2.3456      & 2.3456      & 2.3456      & 2.3456      \\
%             34.2345     & 34.2345     & 34.2345     & 34.2345     \\
%             56.7835     & 56.7835     & 56.7835     & 56.7835     \\
%             90.473      & 90.473      & 90.473      & 90.473      \\
%             \bottomrule
%         \end{tabular}
%     \end{threeparttable}
% \end{table}

% \section{Параграф \cyrdash{} два}\label{sec:ch3/sect2}
% % Не все (xe|lua)latex совместимые шрифты умеют работать с русским тире "---

% Некоторый текст.

% \section{Параграф с подпараграфами}\label{sec:ch3/sect3}

% \subsection{Подпараграф \cyrdash{} один}\label{subsec:ch3/sect3/sub1}

% Некоторый текст.

% \subsection{Подпараграф \cyrdash{} два}\label{subsec:ch3/sect3/sub2}

% Некоторый текст.

% \clearpage
           % Глава 3
% \chapter*{Заключение}                       % Заголовок
\addcontentsline{toc}{chapter}{Заключение}  % Добавляем его в оглавление

%% Согласно ГОСТ Р 7.0.11-2011:
%% 5.3.3 В заключении диссертации излагают итоги выполненного исследования, рекомендации, перспективы дальнейшей разработки темы.
%% 9.2.3 В заключении автореферата диссертации излагают итоги данного исследования, рекомендации и перспективы дальнейшей разработки темы.
%% Поэтому имеет смысл сделать эту часть общей и загрузить из одного файла в автореферат и в диссертацию:

%% Согласно ГОСТ Р 7.0.11-2011:
%% 5.3.3 В заключении диссертации излагают итоги выполненного исследования, рекомендации, перспективы дальнейшей разработки темы.
%% 9.2.3 В заключении автореферата диссертации излагают итоги данного исследования, рекомендации и перспективы дальнейшей разработки темы.
% \begin{enumerate}
%   \item На основе анализа \ldots
%   \item Численные исследования показали, что \ldots
%   \item Математическое моделирование показало \ldots
%   \item Для выполнения поставленных задач был создан \ldots
% \end{enumerate}


% И какая-нибудь заключающая фраза.

% Последний параграф может включать благодарности.  В заключение автор
% выражает благодарность и большую признательность научному руководителю
% Иванову~И.\,И. за поддержку, помощь, обсуждение результатов и~научное
% руководство. Также автор благодарит Сидорова~А.\,А. и~Петрова~Б.\,Б.
% за помощь в~работе с~образцами, Рабиновича~В.\,В. за предоставленные
% образцы и~обсуждение результатов, Занудятину~Г.\,Г. и авторов шаблона
% *Russian-Phd-LaTeX-Dissertation-Template* за~помощь в оформлении
% диссертации. Автор также благодарит много разных людей
% и~всех, кто сделал настоящую работу автора возможной.
      % Заключение
% \include{Dissertation/acronyms}        % Список сокращений и условных обозначений
% \chapter*{Словарь терминов}             % Заголовок
\addcontentsline{toc}{chapter}{Словарь терминов}  % Добавляем его в оглавление

% \textbf{TeX} : Cистема компьютерной вёрстки, разработанная американским профессором информатики Дональдом Кнутом

% \textbf{панграмма} : Короткий текст, использующий все или почти все буквы алфавита
      % Словарь терминов
% \include{Dissertation/references}      % Список литературы
% \include{Dissertation/lists}           % Списки таблиц и изображений (иллюстративный материал)

% \setcounter{totalchapter}{\value{chapter}} % Подсчёт количества глав

% %%% Настройки для приложений
% \appendix
% % Оформление заголовков приложений ближе к ГОСТ:
% \setlength{\midchapskip}{20pt}
% \renewcommand*{\afterchapternum}{\par\nobreak\vskip \midchapskip}
% \renewcommand\thechapter{\Asbuk{chapter}} % Чтобы приложения русскими буквами нумеровались

% \chapter{Динамика популярности поисковых запросов}\label{app:A}

\begingroup
\centering
\captionsetup[table]{skip=7pt}
\begin{longtable}[h!]{|l|r|r|}
\caption{Данные, полученные с помощью сервиса \url{wordstat.yandex.ru} для запроса \texttt{искусственный интеллект} за период с января 2018 г. по июнь 2024 г. по всему миру на всех устройствах}\label{tab:wordstat_dynamics} \\
\hline
\textbf{Период} & \textbf{Число запросов} & \textbf{Доля от всех запросов, \%} \\ \hline
\endfirsthead%
\caption*{Продолжение таблицы~\thetable} \\
\hline
\textbf{Период} & \textbf{Число запросов} & \textbf{Доля от всех запросов, \%} \\ \hline
\endhead
\hline
\endfoot
\hline
\endlastfoot
\hline
январь 2018 & 77 861 & 0,00101 \\
\hline
февраль 2018 & 76 350 & 0,00099 \\
\hline
март 2018 & 88 583 & 0,00106 \\
\hline
апрель 2018 & 84 798 & 0,00112 \\
\hline
май 2018 & 89 388 & 0,00119 \\
\hline
июнь 2018 & 80 337 & 0,00116 \\
\hline
июль 2018 & 73 254 & 0,00105 \\
\hline
август 2018 & 70 866 & 0,00101 \\
\hline
сентябрь 2018 & 88 001 & 0,00111 \\
\hline
октябрь 2018 & 106 668 & 0,00116 \\
\hline
ноябрь 2018 & 105 809 & 0,00125 \\
\hline
декабрь 2018 & 112 565 & 0,00132 \\
\hline
январь 2019 & 100 929 & 0,00115 \\
\hline
февраль 2019 & 103 687 & 0,00124 \\
\hline
март 2019 & 118 130 & 0,00135 \\
\hline
апрель 2019 & 104 869 & 0,00127 \\
\hline
май 2019 & 110 265 & 0,00140 \\
\hline
июнь 2019 & 90 556 & 0,00125 \\
\hline
июль 2019 & 71 127 & 0,00097 \\
\hline
август 2019 & 69 133 & 0,00095 \\
\hline
сентябрь 2019 & 101 920 & 0,00130 \\
\hline
октябрь 2019 & 114 724 & 0,00130 \\
\hline
ноябрь 2019 & 148 313 & 0,00168 \\
\hline
декабрь 2019 & 133 279 & 0,00149 \\
\hline
январь 2020 & 148 119 & 0,00155 \\
\hline
февраль 2020 & 137 028 & 0,00149 \\
\hline
март 2020 & 159 964 & 0,00157 \\
\hline
апрель 2020 & 150 735 & 0,00156 \\
\hline
май 2020 & 159 017 & 0,00162 \\
\hline
июнь 2020 & 143 746 & 0,00146 \\
\hline
июль 2020 & 107 861 & 0,00123 \\
\hline
август 2020 & 112 489 & 0,00133 \\
\hline
сентябрь 2020 & 171 888 & 0,00158 \\
\hline
октябрь 2020 & 180 468 & 0,00162 \\
\hline
ноябрь 2020 & 208 458 & 0,00187 \\
\hline
декабрь 2020 & 249 751 & 0,00234 \\
\hline
январь 2021 & 215 723 & 0,00202 \\
\hline
февраль 2021 & 226 024 & 0,00218 \\
\hline
март 2021 & 273 892 & 0,00258 \\
\hline
апрель 2021 & 237 194 & 0,00237 \\
\hline
май 2021 & 245 980 & 0,00246 \\
\hline
июнь 2021 & 199 186 & 0,00204 \\
\hline
июль 2021 & 136 168 & 0,00158 \\
\hline
август 2021 & 141 436 & 0,00165 \\
\hline
сентябрь 2021 & 276 752 & 0,00268 \\
\hline
октябрь 2021 & 317 919 & 0,00298 \\
\hline
ноябрь 2021 & 359 644 & 0,00346 \\
\hline
декабрь 2021 & 309 078 & 0,00262 \\
\hline
январь 2022 & 294 694 & 0,00248 \\
\hline
февраль 2022 & 272 580 & 0,00232 \\
\hline
март 2022 & 250 165 & 0,00197 \\
\hline
апрель 2022 & 233 001 & 0,00196 \\
\hline
май 2022 & 225 817 & 0,00190 \\
\hline
июнь 2022 & 260 574 & 0,00244 \\
\hline
июль 2022 & 143 968 & 0,00138 \\
\hline
август 2022 & 171 668 & 0,00160 \\
\hline
сентябрь 2022 & 364 072 & 0,00313 \\
\hline
октябрь 2022 & 372 819 & 0,00303 \\
\hline
ноябрь 2022 & 394 433 & 0,00314 \\
\hline
декабрь 2022 & 442 691 & 0,00354 \\
\hline
январь 2023 & 334 185 & 0,00272 \\
\hline
февраль 2023 & 443 693 & 0,00391 \\
\hline
март 2023 & 582 461 & 0,00471 \\
\hline
апрель 2023 & 579 800 & 0,00505 \\
\hline
май 2023 & 584 509 & 0,00519 \\
\hline
июнь 2023 & 476 506 & 0,00467 \\
\hline
июль 2023 & 403 379 & 0,00402 \\
\hline
август 2023 & 407 155 & 0,00408 \\
\hline
сентябрь 2023 & 899 284 & 0,00841 \\
\hline
октябрь 2023 & 961 673 & 0,00811 \\
\hline
ноябрь 2023 & 825 129 & 0,00707 \\
\hline
декабрь 2023 & 934 004 & 0,00787 \\
\hline
январь 2024 & 738 277 & 0,00616 \\
\hline
февраль 2024 & 798 270 & 0,00715 \\
\hline
март 2024 & 845 298 & 0,00737 \\
\hline
апрель 2024 & 787 383 & 0,00730 \\
\hline
май 2024 & 741 390 & 0,00700 \\
\hline
июнь 2024 & 544 455 & 0,00584 \\
\hline
\end{longtable}
\endgroup



% \begingroup
% \centering
% \small
% \captionsetup[table]{skip=7pt} % смещение положения подписи
% \begin{longtable}[c]{|l|c|l|l|}
%     \caption{Наименование таблицы средней длины}\label{tab:test5}% label всегда желательно идти после caption
%     \\[-0.45\onelineskip]
%     \hline
%     Параметр & Умолч. & Тип & Описание                                          \\ \hline
%     \endfirsthead%
%     \caption*{Продолжение таблицы~\thetable}                                    \\[-0.45\onelineskip]
%     \hline
%     Параметр & Умолч. & Тип & Описание                                          \\ \hline
%     \endhead
%     \hline
%     \endfoot
%     \hline
%     \endlastfoot
%     \multicolumn{4}{|l|}{\&INP}                                                 \\ \hline
%     kick     & 1      & int & 0: инициализация без шума (\(p_s = const\))       \\
%              &        &     & 1: генерация белого шума                          \\
%              &        &     & 2: генерация белого шума симметрично относительно \\
%              &        &     & экватора                                          \\
%     mars     & 0      & int & 1: инициализация модели для планеты Марс          \\
%     kick     & 1      & int & 0: инициализация без шума (\(p_s = const\))       \\
%              &        &     & 1: генерация белого шума                          \\
%              &        &     & 2: генерация белого шума симметрично относительно \\
%              &        &     & экватора                                          \\
%     mars     & 0      & int & 1: инициализация модели для планеты Марс          \\
%     kick     & 1      & int & 0: инициализация без шума (\(p_s = const\))       \\
%              &        &     & 1: генерация белого шума                          \\
%              &        &     & 2: генерация белого шума симметрично относительно \\
%              &        &     & экватора                                          \\
%     mars     & 0      & int & 1: инициализация модели для планеты Марс          \\
%     kick     & 1      & int & 0: инициализация без шума (\(p_s = const\))       \\
%              &        &     & 1: генерация белого шума                          \\
%              &        &     & 2: генерация белого шума симметрично относительно \\
%              &        &     & экватора                                          \\
%     mars     & 0      & int & 1: инициализация модели для планеты Марс          \\
%     kick     & 1      & int & 0: инициализация без шума (\(p_s = const\))       \\
%              &        &     & 1: генерация белого шума                          \\
%              &        &     & 2: генерация белого шума симметрично относительно \\
%              &        &     & экватора                                          \\
%     mars     & 0      & int & 1: инициализация модели для планеты Марс          \\
%     kick     & 1      & int & 0: инициализация без шума (\(p_s = const\))       \\
%              &        &     & 1: генерация белого шума                          \\
%              &        &     & 2: генерация белого шума симметрично относительно \\
%              &        &     & экватора                                          \\
%     mars     & 0      & int & 1: инициализация модели для планеты Марс          \\
%     kick     & 1      & int & 0: инициализация без шума (\(p_s = const\))       \\
%              &        &     & 1: генерация белого шума                          \\
%              &        &     & 2: генерация белого шума симметрично относительно \\
%              &        &     & экватора                                          \\
%     mars     & 0      & int & 1: инициализация модели для планеты Марс          \\
%     kick     & 1      & int & 0: инициализация без шума (\(p_s = const\))       \\
%              &        &     & 1: генерация белого шума                          \\
%              &        &     & 2: генерация белого шума симметрично относительно \\
%              &        &     & экватора                                          \\
%     mars     & 0      & int & 1: инициализация модели для планеты Марс          \\
%     kick     & 1      & int & 0: инициализация без шума (\(p_s = const\))       \\
%              &        &     & 1: генерация белого шума                          \\
%              &        &     & 2: генерация белого шума симметрично относительно \\
%              &        &     & экватора                                          \\
%     mars     & 0      & int & 1: инициализация модели для планеты Марс          \\
%     kick     & 1      & int & 0: инициализация без шума (\(p_s = const\))       \\
%              &        &     & 1: генерация белого шума                          \\
%              &        &     & 2: генерация белого шума симметрично относительно \\
%              &        &     & экватора                                          \\
%     mars     & 0      & int & 1: инициализация модели для планеты Марс          \\
%     kick     & 1      & int & 0: инициализация без шума (\(p_s = const\))       \\
%              &        &     & 1: генерация белого шума                          \\
%              &        &     & 2: генерация белого шума симметрично относительно \\
%              &        &     & экватора                                          \\
%     mars     & 0      & int & 1: инициализация модели для планеты Марс          \\
%     kick     & 1      & int & 0: инициализация без шума (\(p_s = const\))       \\
%              &        &     & 1: генерация белого шума                          \\
%              &        &     & 2: генерация белого шума симметрично относительно \\
%              &        &     & экватора                                          \\
%     mars     & 0      & int & 1: инициализация модели для планеты Марс          \\
%     kick     & 1      & int & 0: инициализация без шума (\(p_s = const\))       \\
%              &        &     & 1: генерация белого шума                          \\
%              &        &     & 2: генерация белого шума симметрично относительно \\
%              &        &     & экватора                                          \\
%     mars     & 0      & int & 1: инициализация модели для планеты Марс          \\
%     kick     & 1      & int & 0: инициализация без шума (\(p_s = const\))       \\
%              &        &     & 1: генерация белого шума                          \\
%              &        &     & 2: генерация белого шума симметрично относительно \\
%              &        &     & экватора                                          \\
%     mars     & 0      & int & 1: инициализация модели для планеты Марс          \\
%     kick     & 1      & int & 0: инициализация без шума (\(p_s = const\))       \\
%              &        &     & 1: генерация белого шума                          \\
%              &        &     & 2: генерация белого шума симметрично относительно \\
%              &        &     & экватора                                          \\
%     mars     & 0      & int & 1: инициализация модели для планеты Марс          \\
%     \hline
%     %& & & $\:$ \\
%     \multicolumn{4}{|l|}{\&SURFPAR}                                             \\ \hline
%     kick     & 1      & int & 0: инициализация без шума (\(p_s = const\))       \\
%              &        &     & 1: генерация белого шума                          \\
%              &        &     & 2: генерация белого шума симметрично относительно \\
%              &        &     & экватора                                          \\
%     mars     & 0      & int & 1: инициализация модели для планеты Марс          \\
%     kick     & 1      & int & 0: инициализация без шума (\(p_s = const\))       \\
%              &        &     & 1: генерация белого шума                          \\
%              &        &     & 2: генерация белого шума симметрично относительно \\
%              &        &     & экватора                                          \\
%     mars     & 0      & int & 1: инициализация модели для планеты Марс          \\
%     kick     & 1      & int & 0: инициализация без шума (\(p_s = const\))       \\
%              &        &     & 1: генерация белого шума                          \\
%              &        &     & 2: генерация белого шума симметрично относительно \\
%              &        &     & экватора                                          \\
%     mars     & 0      & int & 1: инициализация модели для планеты Марс          \\
%     kick     & 1      & int & 0: инициализация без шума (\(p_s = const\))       \\
%              &        &     & 1: генерация белого шума                          \\
%              &        &     & 2: генерация белого шума симметрично относительно \\
%              &        &     & экватора                                          \\
%     mars     & 0      & int & 1: инициализация модели для планеты Марс          \\
%     kick     & 1      & int & 0: инициализация без шума (\(p_s = const\))       \\
%              &        &     & 1: генерация белого шума                          \\
%              &        &     & 2: генерация белого шума симметрично относительно \\
%              &        &     & экватора                                          \\
%     mars     & 0      & int & 1: инициализация модели для планеты Марс          \\
%     kick     & 1      & int & 0: инициализация без шума (\(p_s = const\))       \\
%              &        &     & 1: генерация белого шума                          \\
%              &        &     & 2: генерация белого шума симметрично относительно \\
%              &        &     & экватора                                          \\
%     mars     & 0      & int & 1: инициализация модели для планеты Марс          \\
%     kick     & 1      & int & 0: инициализация без шума (\(p_s = const\))       \\
%              &        &     & 1: генерация белого шума                          \\
%              &        &     & 2: генерация белого шума симметрично относительно \\
%              &        &     & экватора                                          \\
%     mars     & 0      & int & 1: инициализация модели для планеты Марс          \\
%     kick     & 1      & int & 0: инициализация без шума (\(p_s = const\))       \\
%              &        &     & 1: генерация белого шума                          \\
%              &        &     & 2: генерация белого шума симметрично относительно \\
%              &        &     & экватора                                          \\
%     mars     & 0      & int & 1: инициализация модели для планеты Марс          \\
%     kick     & 1      & int & 0: инициализация без шума (\(p_s = const\))       \\
%              &        &     & 1: генерация белого шума                          \\
%              &        &     & 2: генерация белого шума симметрично относительно \\
%              &        &     & экватора                                          \\
%     mars     & 0      & int & 1: инициализация модели для планеты Марс          \\
% \end{longtable}
% \normalsize% возвращаем шрифт к нормальному
% \endgroup



% \chapter{Примеры вставки листингов программного кода}\label{app:A}

% Для крупных листингов есть два способа. Первый красивый, но в нём могут быть
% проблемы с поддержкой кириллицы (у вас может встречаться в~комментариях
% и~печатаемых сообщениях), он представлен на листинге~\cref{lst:hwbeauty}.
% \begin{ListingEnv}[!h]% настройки floating аналогичны окружению figure
%     \captiondelim{ } % разделитель идентификатора с номером от наименования
%     \caption{Программа ,,Hello, world`` на \protect\cpp}\label{lst:hwbeauty}
%     % окружение учитывает пробелы и табуляции и применяет их в сответсвии с настройками
%     \begin{lstlisting}[language={[ISO]C++}]
% 	#include <iostream>
% 	using namespace std;

% 	int main() //кириллица в комментариях при xelatex и lualatex имеет проблемы с пробелами
% 	{
% 		cout << "Hello, world" << endl; //latin letters in commentaries
% 		system("pause");
% 		return 0;
% 	}
%     \end{lstlisting}
% \end{ListingEnv}%
% Второй не~такой красивый, но без ограничений (см.~листинг~\cref{lst:hwplain}).
% \begin{ListingEnv}[!h]
%     \captiondelim{ } % разделитель идентификатора с номером от наименования
%     \caption{Программа ,,Hello, world`` без подсветки}\label{lst:hwplain}
%     \begin{Verb}

%         #include <iostream>
%         using namespace std;

%         int main() //кириллица в комментариях
%         {
%             cout << "Привет, мир" << endl;
%         }
%     \end{Verb}
% \end{ListingEnv}

% Можно использовать первый для вставки небольших фрагментов
% внутри текста, а второй для вставки полного
% кода в приложении, если таковое имеется.

% Если нужно вставить совсем короткий пример кода (одна или две строки),
% то~выделение  линейками и нумерация может смотреться чересчур громоздко.
% В таких случаях можно использовать окружения \texttt{lstlisting} или
% \texttt{Verb} без \texttt{ListingEnv}. Приведём такой пример
% с указанием языка программирования, отличного от~заданного по умолчанию:
% \begin{lstlisting}[language=Haskell]
% fibs = 0 : 1 : zipWith (+) fibs (tail fibs)
% \end{lstlisting}
% Такое решение "--- со вставкой нумерованных листингов покрупнее
% и~вставок без выделения для маленьких фрагментов "--- выбрано,
% например, в~книге Эндрю Таненбаума и Тодда Остина по архитектуре
% компьютера.

% Наконец, для оформления идентификаторов внутри строк
% (функция \lstinline{main} и~тому подобное) используется
% \texttt{lstinline} или, самое простое, моноширинный текст
% (\texttt{\textbackslash texttt}).

% Пример~\cref{lst:internal3}, иллюстрирующий подключение переопределённого
% языка. Может быть полезным, если подсветка кода работает криво. Без
% дополнительного окружения, с подписью и ссылкой, реализованной встроенным
% средством.
% \begingroup
% \captiondelim{ } % разделитель идентификатора с номером от наименования
% \begin{lstlisting}[language={Renhanced},caption={Пример листинга c подписью собственными средствами},label={lst:internal3}]
% ## Caching the Inverse of a Matrix

% ## Matrix inversion is usually a costly computation and there may be some
% ## benefit to caching the inverse of a matrix rather than compute it repeatedly
% ## This is a pair of functions that cache the inverse of a matrix.

% ## makeCacheMatrix creates a special "matrix" object that can cache its inverse

% makeCacheMatrix <- function(x = matrix()) {#кириллица в комментариях при xelatex и lualatex имеет проблемы с пробелами
%     i <- NULL
%     set <- function(y) {
%         x <<- y
%         i <<- NULL
%     }
%     get <- function() x
%     setSolved <- function(solve) i <<- solve
%     getSolved <- function() i
%     list(set = set, get = get,
%     setSolved = setSolved,
%     getSolved = getSolved)

% }


% ## cacheSolve computes the inverse of the special "matrix" returned by
% ## makeCacheMatrix above. If the inverse has already been calculated (and the
% ## matrix has not changed), then the cachesolve should retrieve the inverse from
% ## the cache.

% cacheSolve <- function(x, ...) {
%     ## Return a matrix that is the inverse of 'x'
%     i <- x$getSolved()
%     if(!is.null(i)) {
%         message("getting cached data")
%         return(i)
%     }
%     data <- x$get()
%     i <- solve(data, ...)
%     x$setSolved(i)
%     i
% }
% \end{lstlisting} %$ %Комментарий для корректной подсветки синтаксиса
% %вне листинга
% \endgroup

% Листинг~\cref{lst:external1} подгружается из внешнего файла. Приходится
% загружать без окружения дополнительного. Иначе по страницам не переносится.
% \begingroup
% \captiondelim{ } % разделитель идентификатора с номером от наименования
% \lstinputlisting[lastline=78,language={R},caption={Листинг из внешнего файла},label={lst:external1}]{listings/run_analysis.R}
% \endgroup

% \chapter{Очень длинное название второго приложения, в~котором продемонстрирована работа с~длинными таблицами}\label{app:B}

% \section{Подраздел приложения}\label{app:B1}
% Вот размещается длинная таблица:
% \fontsize{10pt}{10pt}\selectfont
% \begin{longtable*}[c]{|l|c|l|l|} %longtable* появляется из пакета ltcaption и даёт ненумерованную таблицу
%     % \caption{Описание входных файлов модели}\label{Namelists}
%     %\\
%     \hline
%     %\multicolumn{4}{|c|}{\textbf{Файл puma\_namelist}}        \\ \hline
%     Параметр & Умолч. & Тип & Описание               \\ \hline
%     \endfirsthead   \hline
%     \multicolumn{4}{|c|}{\small\slshape (продолжение)}        \\ \hline
%     Параметр & Умолч. & Тип & Описание               \\ \hline
%     \endhead        \hline
%     % \multicolumn{4}{|c|}{\small\slshape (окончание)}        \\ \hline
%     % Параметр & Умолч. & Тип & Описание               \\ \hline
%     %                                             \endlasthead        \hline
%     \multicolumn{4}{|r|}{\small\slshape продолжение следует}  \\ \hline
%     \endfoot        \hline
%     \endlastfoot
%     \multicolumn{4}{|l|}{\&INP}        \\ \hline
%     kick & 1 & int & 0: инициализация без шума (\(p_s = const\)) \\
%     &   &     & 1: генерация белого шума                  \\
%     &   &     & 2: генерация белого шума симметрично относительно \\
%     & & & экватора    \\
%     mars & 0 & int & 1: инициализация модели для планеты Марс     \\
%     kick & 1 & int & 0: инициализация без шума (\(p_s = const\)) \\
%     &   &     & 1: генерация белого шума                  \\
%     &   &     & 2: генерация белого шума симметрично относительно \\
%     & & & экватора    \\
%     mars & 0 & int & 1: инициализация модели для планеты Марс     \\
%     kick & 1 & int & 0: инициализация без шума (\(p_s = const\)) \\
%     &   &     & 1: генерация белого шума                  \\
%     &   &     & 2: генерация белого шума симметрично относительно \\
%     & & & экватора    \\
%     mars & 0 & int & 1: инициализация модели для планеты Марс     \\
%     kick & 1 & int & 0: инициализация без шума (\(p_s = const\)) \\
%     &   &     & 1: генерация белого шума                  \\
%     &   &     & 2: генерация белого шума симметрично относительно \\
%     & & & экватора    \\
%     mars & 0 & int & 1: инициализация модели для планеты Марс     \\
%     kick & 1 & int & 0: инициализация без шума (\(p_s = const\)) \\
%     &   &     & 1: генерация белого шума                  \\
%     &   &     & 2: генерация белого шума симметрично относительно \\
%     & & & экватора    \\
%     mars & 0 & int & 1: инициализация модели для планеты Марс     \\
%     kick & 1 & int & 0: инициализация без шума (\(p_s = const\)) \\
%     &   &     & 1: генерация белого шума                  \\
%     &   &     & 2: генерация белого шума симметрично относительно \\
%     & & & экватора    \\
%     mars & 0 & int & 1: инициализация модели для планеты Марс     \\
%     kick & 1 & int & 0: инициализация без шума (\(p_s = const\)) \\
%     &   &     & 1: генерация белого шума                  \\
%     &   &     & 2: генерация белого шума симметрично относительно \\
%     & & & экватора    \\
%     mars & 0 & int & 1: инициализация модели для планеты Марс     \\
%     kick & 1 & int & 0: инициализация без шума (\(p_s = const\)) \\
%     &   &     & 1: генерация белого шума                  \\
%     &   &     & 2: генерация белого шума симметрично относительно \\
%     & & & экватора    \\
%     mars & 0 & int & 1: инициализация модели для планеты Марс     \\
%     kick & 1 & int & 0: инициализация без шума (\(p_s = const\)) \\
%     &   &     & 1: генерация белого шума                  \\
%     &   &     & 2: генерация белого шума симметрично относительно \\
%     & & & экватора    \\
%     mars & 0 & int & 1: инициализация модели для планеты Марс     \\
%     kick & 1 & int & 0: инициализация без шума (\(p_s = const\)) \\
%     &   &     & 1: генерация белого шума                  \\
%     &   &     & 2: генерация белого шума симметрично относительно \\
%     & & & экватора    \\
%     mars & 0 & int & 1: инициализация модели для планеты Марс     \\
%     kick & 1 & int & 0: инициализация без шума (\(p_s = const\)) \\
%     &   &     & 1: генерация белого шума                  \\
%     &   &     & 2: генерация белого шума симметрично относительно \\
%     & & & экватора    \\
%     mars & 0 & int & 1: инициализация модели для планеты Марс     \\
%     kick & 1 & int & 0: инициализация без шума (\(p_s = const\)) \\
%     &   &     & 1: генерация белого шума                  \\
%     &   &     & 2: генерация белого шума симметрично относительно \\
%     & & & экватора    \\
%     mars & 0 & int & 1: инициализация модели для планеты Марс     \\
%     kick & 1 & int & 0: инициализация без шума (\(p_s = const\)) \\
%     &   &     & 1: генерация белого шума                  \\
%     &   &     & 2: генерация белого шума симметрично относительно \\
%     & & & экватора    \\
%     mars & 0 & int & 1: инициализация модели для планеты Марс     \\
%     kick & 1 & int & 0: инициализация без шума (\(p_s = const\)) \\
%     &   &     & 1: генерация белого шума                  \\
%     &   &     & 2: генерация белого шума симметрично относительно \\
%     & & & экватора    \\
%     mars & 0 & int & 1: инициализация модели для планеты Марс     \\
%     kick & 1 & int & 0: инициализация без шума (\(p_s = const\)) \\
%     &   &     & 1: генерация белого шума                  \\
%     &   &     & 2: генерация белого шума симметрично относительно \\
%     & & & экватора    \\
%     mars & 0 & int & 1: инициализация модели для планеты Марс     \\
%     \hline
%     %& & & \(\:\) \\
%     \multicolumn{4}{|l|}{\&SURFPAR}        \\ \hline
%     kick & 1 & int & 0: инициализация без шума (\(p_s = const\)) \\
%     &   &     & 1: генерация белого шума                  \\
%     &   &     & 2: генерация белого шума симметрично относительно \\
%     & & & экватора    \\
%     mars & 0 & int & 1: инициализация модели для планеты Марс     \\
%     kick & 1 & int & 0: инициализация без шума (\(p_s = const\)) \\
%     &   &     & 1: генерация белого шума                  \\
%     &   &     & 2: генерация белого шума симметрично относительно \\
%     & & & экватора    \\
%     mars & 0 & int & 1: инициализация модели для планеты Марс     \\
%     kick & 1 & int & 0: инициализация без шума (\(p_s = const\)) \\
%     &   &     & 1: генерация белого шума                  \\
%     &   &     & 2: генерация белого шума симметрично относительно \\
%     & & & экватора    \\
%     mars & 0 & int & 1: инициализация модели для планеты Марс     \\
%     kick & 1 & int & 0: инициализация без шума (\(p_s = const\)) \\
%     &   &     & 1: генерация белого шума                  \\
%     &   &     & 2: генерация белого шума симметрично относительно \\
%     & & & экватора    \\
%     mars & 0 & int & 1: инициализация модели для планеты Марс     \\
%     kick & 1 & int & 0: инициализация без шума (\(p_s = const\)) \\
%     &   &     & 1: генерация белого шума                  \\
%     &   &     & 2: генерация белого шума симметрично относительно \\
%     & & & экватора    \\
%     mars & 0 & int & 1: инициализация модели для планеты Марс     \\
%     kick & 1 & int & 0: инициализация без шума (\(p_s = const\)) \\
%     &   &     & 1: генерация белого шума                  \\
%     &   &     & 2: генерация белого шума симметрично относительно \\
%     & & & экватора    \\
%     mars & 0 & int & 1: инициализация модели для планеты Марс     \\
%     kick & 1 & int & 0: инициализация без шума (\(p_s = const\)) \\
%     &   &     & 1: генерация белого шума                  \\
%     &   &     & 2: генерация белого шума симметрично относительно \\
%     & & & экватора    \\
%     mars & 0 & int & 1: инициализация модели для планеты Марс     \\
%     kick & 1 & int & 0: инициализация без шума (\(p_s = const\)) \\
%     &   &     & 1: генерация белого шума                  \\
%     &   &     & 2: генерация белого шума симметрично относительно \\
%     & & & экватора    \\
%     mars & 0 & int & 1: инициализация модели для планеты Марс     \\
%     kick & 1 & int & 0: инициализация без шума (\(p_s = const\)) \\
%     &   &     & 1: генерация белого шума                  \\
%     &   &     & 2: генерация белого шума симметрично относительно \\
%     & & & экватора    \\
%     mars & 0 & int & 1: инициализация модели для планеты Марс     \\
%     \hline
% \end{longtable*}

% \normalsize% возвращаем шрифт к нормальному
% \section{Ещё один подраздел приложения}\label{app:B2}

% Нужно больше подразделов приложения!
% Конвынёры витюпырата но нам, тебиквюэ мэнтётюм позтюлант ед про. Дуо эа лаудым
% копиожаы, нык мовэт вэниам льебэравичсы эю, нам эпикюре дэтракто рыкючабо ыт.

% Пример длинной таблицы с записью продолжения по ГОСТ 2.105:

% \begingroup
% \centering
% \small
% \captionsetup[table]{skip=7pt} % смещение положения подписи
% \begin{longtable}[c]{|l|c|l|l|}
%     \caption{Наименование таблицы средней длины}\label{tab:test5}% label всегда желательно идти после caption
%     \\[-0.45\onelineskip]
%     \hline
%     Параметр & Умолч. & Тип & Описание                                          \\ \hline
%     \endfirsthead%
%     \caption*{Продолжение таблицы~\thetable}                                    \\[-0.45\onelineskip]
%     \hline
%     Параметр & Умолч. & Тип & Описание                                          \\ \hline
%     \endhead
%     \hline
%     \endfoot
%     \hline
%     \endlastfoot
%     \multicolumn{4}{|l|}{\&INP}                                                 \\ \hline
%     kick     & 1      & int & 0: инициализация без шума (\(p_s = const\))       \\
%              &        &     & 1: генерация белого шума                          \\
%              &        &     & 2: генерация белого шума симметрично относительно \\
%              &        &     & экватора                                          \\
%     mars     & 0      & int & 1: инициализация модели для планеты Марс          \\
%     kick     & 1      & int & 0: инициализация без шума (\(p_s = const\))       \\
%              &        &     & 1: генерация белого шума                          \\
%              &        &     & 2: генерация белого шума симметрично относительно \\
%              &        &     & экватора                                          \\
%     mars     & 0      & int & 1: инициализация модели для планеты Марс          \\
%     kick     & 1      & int & 0: инициализация без шума (\(p_s = const\))       \\
%              &        &     & 1: генерация белого шума                          \\
%              &        &     & 2: генерация белого шума симметрично относительно \\
%              &        &     & экватора                                          \\
%     mars     & 0      & int & 1: инициализация модели для планеты Марс          \\
%     kick     & 1      & int & 0: инициализация без шума (\(p_s = const\))       \\
%              &        &     & 1: генерация белого шума                          \\
%              &        &     & 2: генерация белого шума симметрично относительно \\
%              &        &     & экватора                                          \\
%     mars     & 0      & int & 1: инициализация модели для планеты Марс          \\
%     kick     & 1      & int & 0: инициализация без шума (\(p_s = const\))       \\
%              &        &     & 1: генерация белого шума                          \\
%              &        &     & 2: генерация белого шума симметрично относительно \\
%              &        &     & экватора                                          \\
%     mars     & 0      & int & 1: инициализация модели для планеты Марс          \\
%     kick     & 1      & int & 0: инициализация без шума (\(p_s = const\))       \\
%              &        &     & 1: генерация белого шума                          \\
%              &        &     & 2: генерация белого шума симметрично относительно \\
%              &        &     & экватора                                          \\
%     mars     & 0      & int & 1: инициализация модели для планеты Марс          \\
%     kick     & 1      & int & 0: инициализация без шума (\(p_s = const\))       \\
%              &        &     & 1: генерация белого шума                          \\
%              &        &     & 2: генерация белого шума симметрично относительно \\
%              &        &     & экватора                                          \\
%     mars     & 0      & int & 1: инициализация модели для планеты Марс          \\
%     kick     & 1      & int & 0: инициализация без шума (\(p_s = const\))       \\
%              &        &     & 1: генерация белого шума                          \\
%              &        &     & 2: генерация белого шума симметрично относительно \\
%              &        &     & экватора                                          \\
%     mars     & 0      & int & 1: инициализация модели для планеты Марс          \\
%     kick     & 1      & int & 0: инициализация без шума (\(p_s = const\))       \\
%              &        &     & 1: генерация белого шума                          \\
%              &        &     & 2: генерация белого шума симметрично относительно \\
%              &        &     & экватора                                          \\
%     mars     & 0      & int & 1: инициализация модели для планеты Марс          \\
%     kick     & 1      & int & 0: инициализация без шума (\(p_s = const\))       \\
%              &        &     & 1: генерация белого шума                          \\
%              &        &     & 2: генерация белого шума симметрично относительно \\
%              &        &     & экватора                                          \\
%     mars     & 0      & int & 1: инициализация модели для планеты Марс          \\
%     kick     & 1      & int & 0: инициализация без шума (\(p_s = const\))       \\
%              &        &     & 1: генерация белого шума                          \\
%              &        &     & 2: генерация белого шума симметрично относительно \\
%              &        &     & экватора                                          \\
%     mars     & 0      & int & 1: инициализация модели для планеты Марс          \\
%     kick     & 1      & int & 0: инициализация без шума (\(p_s = const\))       \\
%              &        &     & 1: генерация белого шума                          \\
%              &        &     & 2: генерация белого шума симметрично относительно \\
%              &        &     & экватора                                          \\
%     mars     & 0      & int & 1: инициализация модели для планеты Марс          \\
%     kick     & 1      & int & 0: инициализация без шума (\(p_s = const\))       \\
%              &        &     & 1: генерация белого шума                          \\
%              &        &     & 2: генерация белого шума симметрично относительно \\
%              &        &     & экватора                                          \\
%     mars     & 0      & int & 1: инициализация модели для планеты Марс          \\
%     kick     & 1      & int & 0: инициализация без шума (\(p_s = const\))       \\
%              &        &     & 1: генерация белого шума                          \\
%              &        &     & 2: генерация белого шума симметрично относительно \\
%              &        &     & экватора                                          \\
%     mars     & 0      & int & 1: инициализация модели для планеты Марс          \\
%     kick     & 1      & int & 0: инициализация без шума (\(p_s = const\))       \\
%              &        &     & 1: генерация белого шума                          \\
%              &        &     & 2: генерация белого шума симметрично относительно \\
%              &        &     & экватора                                          \\
%     mars     & 0      & int & 1: инициализация модели для планеты Марс          \\
%     \hline
%     %& & & $\:$ \\
%     \multicolumn{4}{|l|}{\&SURFPAR}                                             \\ \hline
%     kick     & 1      & int & 0: инициализация без шума (\(p_s = const\))       \\
%              &        &     & 1: генерация белого шума                          \\
%              &        &     & 2: генерация белого шума симметрично относительно \\
%              &        &     & экватора                                          \\
%     mars     & 0      & int & 1: инициализация модели для планеты Марс          \\
%     kick     & 1      & int & 0: инициализация без шума (\(p_s = const\))       \\
%              &        &     & 1: генерация белого шума                          \\
%              &        &     & 2: генерация белого шума симметрично относительно \\
%              &        &     & экватора                                          \\
%     mars     & 0      & int & 1: инициализация модели для планеты Марс          \\
%     kick     & 1      & int & 0: инициализация без шума (\(p_s = const\))       \\
%              &        &     & 1: генерация белого шума                          \\
%              &        &     & 2: генерация белого шума симметрично относительно \\
%              &        &     & экватора                                          \\
%     mars     & 0      & int & 1: инициализация модели для планеты Марс          \\
%     kick     & 1      & int & 0: инициализация без шума (\(p_s = const\))       \\
%              &        &     & 1: генерация белого шума                          \\
%              &        &     & 2: генерация белого шума симметрично относительно \\
%              &        &     & экватора                                          \\
%     mars     & 0      & int & 1: инициализация модели для планеты Марс          \\
%     kick     & 1      & int & 0: инициализация без шума (\(p_s = const\))       \\
%              &        &     & 1: генерация белого шума                          \\
%              &        &     & 2: генерация белого шума симметрично относительно \\
%              &        &     & экватора                                          \\
%     mars     & 0      & int & 1: инициализация модели для планеты Марс          \\
%     kick     & 1      & int & 0: инициализация без шума (\(p_s = const\))       \\
%              &        &     & 1: генерация белого шума                          \\
%              &        &     & 2: генерация белого шума симметрично относительно \\
%              &        &     & экватора                                          \\
%     mars     & 0      & int & 1: инициализация модели для планеты Марс          \\
%     kick     & 1      & int & 0: инициализация без шума (\(p_s = const\))       \\
%              &        &     & 1: генерация белого шума                          \\
%              &        &     & 2: генерация белого шума симметрично относительно \\
%              &        &     & экватора                                          \\
%     mars     & 0      & int & 1: инициализация модели для планеты Марс          \\
%     kick     & 1      & int & 0: инициализация без шума (\(p_s = const\))       \\
%              &        &     & 1: генерация белого шума                          \\
%              &        &     & 2: генерация белого шума симметрично относительно \\
%              &        &     & экватора                                          \\
%     mars     & 0      & int & 1: инициализация модели для планеты Марс          \\
%     kick     & 1      & int & 0: инициализация без шума (\(p_s = const\))       \\
%              &        &     & 1: генерация белого шума                          \\
%              &        &     & 2: генерация белого шума симметрично относительно \\
%              &        &     & экватора                                          \\
%     mars     & 0      & int & 1: инициализация модели для планеты Марс          \\
% \end{longtable}
% \normalsize% возвращаем шрифт к нормальному
% \endgroup
% \section{Использование длинных таблиц с окружением \textit{longtabu}}\label{app:B2a}

% В таблице \cref{tab:test-functions} более книжный вариант
% длинной таблицы, используя окружение \verb!longtabu! и разнообразные
% \verb!toprule! \verb!midrule! \verb!bottomrule! из~пакета
% \verb!booktabs!. Чтобы визуально таблица смотрелась лучше, можно
% использовать следующие параметры: в самом начале задаётся расстояние
% между строчками с~помощью \verb!arraystretch!. Таблица задаётся на
% всю ширину, \verb!longtabu! позволяет делить ширину колонок
% пропорционально "--- тут три колонки в~пропорции 1.1:1:4 "--- для каждой
% колонки первый параметр в~описании \verb!X[]!. Кроме того, в~таблице
% убраны отступы слева и справа с~помощью \verb!@{}!
% в~преамбуле таблицы. К~первому и~второму столбцу применяется
% модификатор

% \verb!>{\setlength{\baselineskip}{0.7\baselineskip}}!,

% \noindent который уменьшает межстрочный интервал в для текста таблиц (иначе
% заголовок второго столбца значительно шире, а двухстрочное имя
% сливается с~окружающими). Для первой и второй колонки текст в ячейках
% выравниваются по~центру как по~вертикали, так и по горизонтали "---
% задаётся буквами \verb!m!~и~\verb!c!~в~описании столбца \verb!X[]!.

% Так как формулы большие "--- используется окружение \verb!alignedat!,
% чтобы отступ был одинаковый у всех формул "--- он сделан для всех, хотя
% для большей части можно было и не использовать.  Чтобы формулы
% занимали поменьше места в~каждом столбце формулы (где надо)
% используется \verb!\textstyle! "--- он~делает дроби меньше, у~знаков
% суммы и произведения "--- индексы сбоку. Иногда формула слишком большая,
% сливается со следующей, поэтому после неё ставится небольшой
% дополнительный отступ \verb!\vspace*{2ex}!. Для штрафных функций "---
% размер фигурных скобок задан вручную \verb!\Big\{!, т.\:к. не~умеет
% \verb!alignedat! работать с~\verb!\left! и~\verb!\right! через
% несколько строк/колонок.

% В примечании к таблице наоборот, окружение \verb!cases! даёт слишком
% большие промежутки между вариантами, чтобы их уменьшить, в конце
% каждой строчки окружения использовался отрицательный дополнительный
% отступ \verb!\\[-0.5em]!.

% \begingroup % Ограничиваем область видимости arraystretch
% \renewcommand{\arraystretch}{1.6}%% Увеличение расстояния между рядами, для улучшения восприятия.
% \begin{longtabu} to \textwidth
%     {%
%     @{}>{\setlength{\baselineskip}{0.7\baselineskip}}X[1.1mc]%
%     >{\setlength{\baselineskip}{0.7\baselineskip}}X[1.1mc]%
%     X[4]@{}%
%     }
%     \caption{Тестовые функции для оптимизации, \(D\) "---
%         размерность. Для всех функций значение в точке глобального
%         минимума равно нулю.\label{tab:test-functions}}\\% label всегда желательно идти после caption

%     \toprule     %%% верхняя линейка
%     Имя           &Стартовый диапазон параметров &Функция  \\
%     \midrule %%% тонкий разделитель. Отделяет названия столбцов. Обязателен по ГОСТ 2.105 пункт 4.4.5
%     \endfirsthead

%     \multicolumn{3}{c}{\small\slshape (продолжение)}        \\
%     \toprule     %%% верхняя линейка
%     Имя           &Стартовый диапазон параметров &Функция  \\
%     \midrule %%% тонкий разделитель. Отделяет названия столбцов. Обязателен по ГОСТ 2.105 пункт 4.4.5
%     \endhead

%     \multicolumn{3}{c}{\small\slshape (окончание)}        \\
%     \toprule     %%% верхняя линейка
%     Имя           &Стартовый диапазон параметров &Функция  \\
%     \midrule %%% тонкий разделитель. Отделяет названия столбцов. Обязателен по ГОСТ 2.105 пункт 4.4.5
%     \endlasthead

%     \bottomrule %%% нижняя линейка
%     \multicolumn{3}{r}{\small\slshape продолжение следует}  \\
%     \endfoot
%     \endlastfoot

%     сфера         &\(\left[-100,\,100\right]^D\)   &
%     \(\begin{aligned}
%         \textstyle f_1(x)=\sum_{i=1}^Dx_i^2
%     \end{aligned}\) \\
%     Schwefel 2.22 &\(\left[-10,\,10\right]^D\)     &
%     \(\begin{aligned}
%         \textstyle f_2(x)=\sum_{i=1}^D|x_i|+\prod_{i=1}^D|x_i|
%     \end{aligned}\) \\
%     Schwefel 1.2  &\(\left[-100,\,100\right]^D\)   &
%     \(\begin{aligned}
%         \textstyle f_3(x)=\sum_{i=1}^D\left(\sum_{j=1}^ix_j\right)^2
%     \end{aligned}\) \\
%     Schwefel 2.21 &\(\left[-100,\,100\right]^D\)   &
%     \(\begin{aligned}
%         \textstyle f_4(x)=\max_i\!\left\{\left|x_i\right|\right\}
%     \end{aligned}\) \\
%     Rosenbrock    &\(\left[-30,\,30\right]^D\)     &
%     \(\begin{aligned}
%         \textstyle f_5(x)=
%         \sum_{i=1}^{D-1}
%         \left[100\!\left(x_{i+1}-x_i^2\right)^2+(x_i-1)^2\right]
%     \end{aligned}\) \\
%     ступенчатая   &\(\left[-100,\,100\right]^D\)   &
%     \(\begin{aligned}
%         \textstyle f_6(x)=\sum_{i=1}^D\big\lfloor x_i+0.5\big\rfloor^2
%     \end{aligned}\) \\
%     зашумлённая квартическая &\(\left[-1.28,\,1.28\right]^D\) &
%     \(\begin{aligned}
%         \textstyle f_7(x)=\sum_{i=1}^Dix_i^4+rand[0,1)
%     \end{aligned}\)\vspace*{2ex}\\
%     Schwefel 2.26 &\(\left[-500,\,500\right]^D\)   &
%     \(\begin{aligned}
%         f_8(x)= & \textstyle\sum_{i=1}^D-x_i\,\sin\sqrt{|x_i|}\,+ \\
%                 & \vphantom{\sum}+ D\cdot
%         418.98288727243369
%     \end{aligned}\)\\
%     Rastrigin     &\(\left[-5.12,\,5.12\right]^D\) &
%     \(\begin{aligned}
%         \textstyle f_9(x)=\sum_{i=1}^D\left[x_i^2-10\,\cos(2\pi x_i)+10\right]
%     \end{aligned}\)\vspace*{2ex}\\
%     Ackley        &\(\left[-32,\,32\right]^D\)     &
%     \(\begin{aligned}
%         f_{10}(x)= & \textstyle -20\, \exp\!\left(
%         -0.2\sqrt{\frac{1}{D}\sum_{i=1}^Dx_i^2} \right)- \\
%                    & \textstyle - \exp\left(
%             \frac{1}{D}\sum_{i=1}^D\cos(2\pi x_i)  \right)
%         + 20 + e
%     \end{aligned}\) \\
%     Griewank      &\(\left[-600,\,600\right]^D\) &
%     \(\begin{aligned}
%         f_{11}(x)= & \textstyle \frac{1}{4000}\sum_{i=1}^{D}x_i^2 -
%         \prod_{i=1}^D\cos\left(x_i/\sqrt{i}\right) +1
%     \end{aligned}\) \vspace*{3ex} \\
%     штрафная 1    &\(\left[-50,\,50\right]^D\)     &
%     \(\begin{aligned}
%         f_{12}(x)= & \textstyle \frac{\pi}{D}\Big\{ 10\,\sin^2(\pi y_1) +            \\
%                    & +\textstyle \sum_{i=1}^{D-1}(y_i-1)^2
%         \left[1+10\,\sin^2(\pi y_{i+1})\right] +                                     \\
%                    & +(y_D-1)^2 \Big\} +\textstyle\sum_{i=1}^D u(x_i,\,10,\,100,\,4)
%     \end{aligned}\) \vspace*{2ex} \\
%     штрафная 2    &\(\left[-50,\,50\right]^D\)     &
%     \(\begin{aligned}
%         f_{13}(x)= & \textstyle 0.1 \Big\{\sin^2(3\pi x_1) +            \\
%                    & +\textstyle \sum_{i=1}^{D-1}(x_i-1)^2
%         \left[1+\sin^2(3 \pi x_{i+1})\right] +                          \\
%                    & +(x_D-1)^2\left[1+\sin^2(2\pi x_D)\right] \Big\} + \\
%                    & +\textstyle\sum_{i=1}^D u(x_i,\,5,\,100,\,4)
%     \end{aligned}\)\\
%     сфера         &\(\left[-100,\,100\right]^D\)   &
%     \(\begin{aligned}
%         \textstyle f_1(x)=\sum_{i=1}^Dx_i^2
%     \end{aligned}\) \\
%     Schwefel 2.22 &\(\left[-10,\,10\right]^D\)     &
%     \(\begin{aligned}
%         \textstyle f_2(x)=\sum_{i=1}^D|x_i|+\prod_{i=1}^D|x_i|
%     \end{aligned}\) \\
%     Schwefel 1.2  &\(\left[-100,\,100\right]^D\)   &
%     \(\begin{aligned}
%         \textstyle f_3(x)=\sum_{i=1}^D\left(\sum_{j=1}^ix_j\right)^2
%     \end{aligned}\) \\
%     Schwefel 2.21 &\(\left[-100,\,100\right]^D\)   &
%     \(\begin{aligned}
%         \textstyle f_4(x)=\max_i\!\left\{\left|x_i\right|\right\}
%     \end{aligned}\) \\
%     Rosenbrock    &\(\left[-30,\,30\right]^D\)     &
%     \(\begin{aligned}
%         \textstyle f_5(x)=
%         \sum_{i=1}^{D-1}
%         \left[100\!\left(x_{i+1}-x_i^2\right)^2+(x_i-1)^2\right]
%     \end{aligned}\) \\
%     ступенчатая   &\(\left[-100,\,100\right]^D\)   &
%     \(\begin{aligned}
%         \textstyle f_6(x)=\sum_{i=1}^D\big\lfloor x_i+0.5\big\rfloor^2
%     \end{aligned}\) \\
%     зашумлённая квартическая &\(\left[-1.28,\,1.28\right]^D\) &
%     \(\begin{aligned}
%         \textstyle f_7(x)=\sum_{i=1}^Dix_i^4+rand[0,1)
%     \end{aligned}\)\vspace*{2ex}\\
%     Schwefel 2.26 &\(\left[-500,\,500\right]^D\)   &
%     \(\begin{aligned}
%         f_8(x)= & \textstyle\sum_{i=1}^D-x_i\,\sin\sqrt{|x_i|}\,+ \\
%                 & \vphantom{\sum}+ D\cdot
%         418.98288727243369
%     \end{aligned}\)\\
%     Rastrigin     &\(\left[-5.12,\,5.12\right]^D\) &
%     \(\begin{aligned}
%         \textstyle f_9(x)=\sum_{i=1}^D\left[x_i^2-10\,\cos(2\pi x_i)+10\right]
%     \end{aligned}\)\vspace*{2ex}\\
%     Ackley        &\(\left[-32,\,32\right]^D\)     &
%     \(\begin{aligned}
%         f_{10}(x)= & \textstyle -20\, \exp\!\left(
%         -0.2\sqrt{\frac{1}{D}\sum_{i=1}^Dx_i^2} \right)- \\
%                    & \textstyle - \exp\left(
%             \frac{1}{D}\sum_{i=1}^D\cos(2\pi x_i)  \right)
%         + 20 + e
%     \end{aligned}\) \\
%     Griewank      &\(\left[-600,\,600\right]^D\) &
%     \(\begin{aligned}
%         f_{11}(x)= & \textstyle \frac{1}{4000}\sum_{i=1}^{D}x_i^2 -
%         \prod_{i=1}^D\cos\left(x_i/\sqrt{i}\right) +1
%     \end{aligned}\) \vspace*{3ex} \\
%     штрафная 1    &\(\left[-50,\,50\right]^D\)     &
%     \(\begin{aligned}
%         f_{12}(x)= & \textstyle \frac{\pi}{D}\Big\{ 10\,\sin^2(\pi y_1) +            \\
%                    & +\textstyle \sum_{i=1}^{D-1}(y_i-1)^2
%         \left[1+10\,\sin^2(\pi y_{i+1})\right] +                                     \\
%                    & +(y_D-1)^2 \Big\} +\textstyle\sum_{i=1}^D u(x_i,\,10,\,100,\,4)
%     \end{aligned}\) \vspace*{2ex} \\
%     штрафная 2    &\(\left[-50,\,50\right]^D\)     &
%     \(\begin{aligned}
%         f_{13}(x)= & \textstyle 0.1 \Big\{\sin^2(3\pi x_1) +            \\
%                    & +\textstyle \sum_{i=1}^{D-1}(x_i-1)^2
%         \left[1+\sin^2(3 \pi x_{i+1})\right] +                          \\
%                    & +(x_D-1)^2\left[1+\sin^2(2\pi x_D)\right] \Big\} + \\
%                    & +\textstyle\sum_{i=1}^D u(x_i,\,5,\,100,\,4)
%     \end{aligned}\)\\
%     \midrule%%% тонкий разделитель
%     \multicolumn{3}{@{}p{\textwidth}}{%
%     \vspace*{-3.5ex}% этим подтягиваем повыше
%     \hspace*{2.5em}% абзацный отступ - требование ГОСТ 2.105
%     Примечание "---  Для функций \(f_{12}\) и \(f_{13}\)
%     используется \(y_i = 1 + \frac{1}{4}(x_i+1)\)
%     и~$u(x_i,\,a,\,k,\,m)=
%         \begin{cases*}
%             k(x_i-a)^m,  & \( x_i >a \)            \\[-0.5em]
%             0,           & \( -a\leq x_i \leq a \) \\[-0.5em]
%             k(-x_i-a)^m, & \( x_i <-a \)
%         \end{cases*}
%     $
%     }\\
%     \bottomrule %%% нижняя линейка
% \end{longtabu}
% \endgroup

% \section{Форматирование внутри таблиц}\label{app:B3}

% В таблице \cref{tab:other-row} пример с чересстрочным
% форматированием. В~файле \verb+userstyles.tex+  задаётся счётчик
% \verb+\newcounter{rowcnt}+ который увеличивается на~1 после каждой
% строчки (как указано в преамбуле таблицы). Кроме того, задаётся
% условный макрос \verb+\altshape+ который выдаёт одно
% из~двух типов форматирования в~зависимости от чётности счётчика.

% В таблице \cref{tab:other-row} каждая чётная строчка "--- синяя,
% нечётная "--- с наклоном и~слегка поднята вверх. Визуально это приводит
% к тому, что среднее значение и~среднеквадратичное изменение
% группируются и хорошо выделяются взглядом в~таблице. Сохраняется
% возможность отдельные значения в таблице выделить цветом или
% шрифтом. К первому и второму столбцу форматирование не применяется
% по~сути таблицы, к шестому общее форматирование не~применяется для
% наглядности.

% Так как заголовок таблицы тоже считается за строчку, то перед ним (для
% первого, промежуточного и финального варианта) счётчик обнуляется,
% а~в~\verb+\altshape+ для нулевого значения счётчика форматирования
% не~применяется.

% \begingroup % Ограничиваем область видимости arraystretch
% \renewcommand\altshape{
%     \ifnumequal{\value{rowcnt}}{0}{
%         % Стиль для заголовка таблицы
%     }{
%         \ifnumodd{\value{rowcnt}}
%         {
%             \color{blue} % Cтиль для нечётных строк
%         }{
%             \vspace*{-0.7ex}\itshape} % Стиль для чётных строк
%     }
% }
% \newcolumntype{A}{>{\centering\begingroup\altshape}X[1mc]<{\endgroup}}
% \needspace{2\baselineskip}
% \renewcommand{\arraystretch}{0.9}%% Уменьшаем  расстояние между
% %% рядами, чтобы таблица не так много
% %% места занимала в дисере.
% \begin{longtabu} to \textwidth {@{}X[0.27ml]@{}X[0.7mc]@{}A@{}A@{}A@{}X[0.98mc]@{}>{\setlength{\baselineskip}{0.7\baselineskip}}A@{}A<{\stepcounter{rowcnt}}@{}}
%     % \begin{longtabu} to \textwidth {@{}X[0.2ml]X[1mc]X[1mc]X[1mc]X[1mc]X[1mc]>{\setlength{\baselineskip}{0.7\baselineskip}}X[1mc]X[1mc]@{}}
%     \caption{Длинная таблица с примером чересстрочного форматирования\label{tab:other-row}}\vspace*{1ex}\\% label всегда желательно идти после caption
%     % \vspace*{1ex}     \\

%     \toprule %%% верхняя линейка
%     \setcounter{rowcnt}{0} &Итера\-ции & JADE\texttt{++} & JADE & jDE & SaDE
%     & DE/rand /1/bin & PSO \\
%     \midrule %%% тонкий разделитель. Отделяет названия столбцов. Обязателен по ГОСТ 2.105 пункт 4.4.5
%     \endfirsthead

%     \multicolumn{8}{c}{\small\slshape (продолжение)} \\
%     \toprule %%% верхняя линейка
%     \setcounter{rowcnt}{0} &Итера\-ции & JADE\texttt{++} & JADE & jDE & SaDE
%     & DE/rand /1/bin & PSO \\
%     \midrule %%% тонкий разделитель. Отделяет названия столбцов. Обязателен по ГОСТ 2.105 пункт 4.4.5
%     \endhead

%     \multicolumn{8}{c}{\small\slshape (окончание)} \\
%     \toprule %%% верхняя линейка
%     \setcounter{rowcnt}{0} &Итера\-ции & JADE\texttt{++} & JADE & jDE & SaDE
%     & DE/rand /1/bin & PSO \\
%     \midrule %%% тонкий разделитель. Отделяет названия столбцов. Обязателен по ГОСТ 2.105 пункт 4.4.5
%     \endlasthead

%     \bottomrule %%% нижняя линейка
%     \multicolumn{8}{r}{\small\slshape продолжение следует}     \\
%     \endfoot
%     \endlastfoot

%     f1  & 1500 & \textbf{1.8E-60}   & 1.3E-54   & 2.5E-28   & 4.5E-20   & 9.8E-14   & 9.6E-42   \\\nopagebreak
%     &      & (8.4E-60) & (9.2E-54) & {\color{red}(3.5E-28)} & (6.9E-20) & (8.4E-14) & (2.7E-41) \\
%     f2  & 2000 & 1.8E-25   & 3.9E-22   & 1.5E-23   & 1.9E-14   & 1.6E-09   & 9.3E-21   \\\nopagebreak
%     &      & (8.8E-25) & (2.7E-21) & (1.0E-23) & (1.1E-14) & (1.1E-09) & (6.3E-20) \\
%     f3  & 5000 & 5.7E-61   & 6.0E-87   & 5.2E-14   & {\color{green}9.0E-37}   & 6.6E-11   & 2.5E-19   \\\nopagebreak
%     &      & (2.7E-60) & (1.9E-86) & (1.1E-13) & (5.4E-36) & (8.8E-11) & (3.9E-19) \\
%     f4  & 5000 & 8.2E-24   & 4.3E-66   & 1.4E-15   & 7.4E-11   & 4.2E-01   & 4.4E-14   \\\nopagebreak
%     &      & (4.0E-23) & (1.2E-65) & (1.0E-15) & (1.8E-10) & (1.1E+00) & (9.3E-14) \\
%     f5  & 3000 & 8.0E-02   & 3.2E-01   & 1.3E+01   & 2.1E+01   & 2.1E+00   & 2.5E+01   \\\nopagebreak
%     &      & (5.6E-01) & (1.1E+00) & (1.4E+01) & (7.8E+00) & (1.5E+00) & (3.2E+01) \\
%     f6  & 100  & 2.9E+00   & 5.6E+00   & 1.0E+03   & 9.3E+02   & 4.7E+03   & 4.5E+01   \\\nopagebreak
%     &      & (1.2E+00) & (1.6E+00) & (2.2E+02) & (1.8E+02) & (1.1E+03) & (2.4E+01) \\
%     f7  & 3000 & 6.4E-04   & 6.8E-04   & 3.3E-03   & 4.8E-03   & 4.7E-03   & 2.5E-03   \\\nopagebreak
%     &      & (2.5E-04) & (2.5E-04) & (8.5E-04) & (1.2E-03) & (1.2E-03) & (1.4E-03) \\
%     f8  & 1000 & 3.3E-05   & 7.1E+00   & 7.9E-11   & 4.7E+00   & 5.9E+03   & 2.4E+03   \\\nopagebreak
%     &      & (2.3E-05) & (2.8E+01) & (1.3E-10) & (3.3E+01) & (1.1E+03) & (6.7E+02) \\
%     f9  & 1000 & 1.0E-04   & 1.4E-04   & 1.5E-04   & 1.2E-03   & 1.8E+02   & 5.2E+01   \\\nopagebreak
%     &      & (6.0E-05) & (6.5E-05) & (2.0E-04) & (6.5E-04) & (1.3E+01) & (1.6E+01) \\
%     f10 & 500  & 8.2E-10   & 3.0E-09   & 3.5E-04   & 2.7E-03   & 1.1E-01   & 4.6E-01   \\\nopagebreak
%     &      & (6.9E-10) & (2.2E-09) & (1.0E-04) & (5.1E-04) & (3.9E-02) & (6.6E-01) \\
%     f11 & 500  & 9.9E-08   & 2.0E-04   & 1.9E-05   & 7.8E-04  & 2.0E-01   & 1.3E-02   \\\nopagebreak
%     &      & (6.0E-07) & (1.4E-03) & (5.8E-05) & (1.2E-03)  & (1.1E-01) & (1.7E-02) \\
%     f12 & 500  & 4.6E-17   & 3.8E-16   & 1.6E-07   & 1.9E-05   & 1.2E-02   & 1.9E-01   \\\nopagebreak
%     &      & (1.9E-16) & (8.3E-16) & (1.5E-07) & (9.2E-06) & (1.0E-02) & (3.9E-01) \\
%     f13 & 500  & 2.0E-16   & 1.2E-15   & 1.5E-06   & 6.1E-05   & 7.5E-02   & 2.9E-03   \\\nopagebreak
%     &      & (6.5E-16) & (2.8E-15) & (9.8E-07) & (2.0E-05) & (3.8E-02) & (4.8E-03) \\
%     f1  & 1500 & \textbf{1.8E-60}   & 1.3E-54   & 2.5E-28   & 4.5E-20   & 9.8E-14   & 9.6E-42   \\\nopagebreak
%     &      & (8.4E-60) & (9.2E-54) & {\color{red}(3.5E-28)} & (6.9E-20) & (8.4E-14) & (2.7E-41) \\
%     f2  & 2000 & 1.8E-25   & 3.9E-22   & 1.5E-23   & 1.9E-14   & 1.6E-09   & 9.3E-21   \\\nopagebreak
%     &      & (8.8E-25) & (2.7E-21) & (1.0E-23) & (1.1E-14) & (1.1E-09) & (6.3E-20) \\
%     f3  & 5000 & 5.7E-61   & 6.0E-87   & 5.2E-14   & 9.0E-37   & 6.6E-11   & 2.5E-19   \\\nopagebreak
%     &      & (2.7E-60) & (1.9E-86) & (1.1E-13) & (5.4E-36) & (8.8E-11) & (3.9E-19) \\
%     f4  & 5000 & 8.2E-24   & 4.3E-66   & 1.4E-15   & 7.4E-11   & 4.2E-01   & 4.4E-14   \\\nopagebreak
%     &      & (4.0E-23) & (1.2E-65) & (1.0E-15) & (1.8E-10) & (1.1E+00) & (9.3E-14) \\
%     f5  & 3000 & 8.0E-02   & 3.2E-01   & 1.3E+01   & 2.1E+01   & 2.1E+00   & 2.5E+01   \\\nopagebreak
%     &      & (5.6E-01) & (1.1E+00) & (1.4E+01) & (7.8E+00) & (1.5E+00) & (3.2E+01) \\
%     f6  & 100  & 2.9E+00   & 5.6E+00   & 1.0E+03   & 9.3E+02   & 4.7E+03   & 4.5E+01   \\\nopagebreak
%     &      & (1.2E+00) & (1.6E+00) & (2.2E+02) & (1.8E+02) & (1.1E+03) & (2.4E+01) \\
%     f7  & 3000 & 6.4E-04   & 6.8E-04   & 3.3E-03   & 4.8E-03   & 4.7E-03   & 2.5E-03   \\\nopagebreak
%     &      & (2.5E-04) & (2.5E-04) & (8.5E-04) & (1.2E-03) & (1.2E-03) & (1.4E-03) \\
%     f8  & 1000 & 3.3E-05   & 7.1E+00   & 7.9E-11   & 4.7E+00   & 5.9E+03   & 2.4E+03   \\\nopagebreak
%     &      & (2.3E-05) & (2.8E+01) & (1.3E-10) & (3.3E+01) & (1.1E+03) & (6.7E+02) \\
%     f9  & 1000 & 1.0E-04   & 1.4E-04   & 1.5E-04   & 1.2E-03   & 1.8E+02   & 5.2E+01   \\\nopagebreak
%     &      & (6.0E-05) & (6.5E-05) & (2.0E-04) & (6.5E-04) & (1.3E+01) & (1.6E+01) \\
%     f10 & 500  & 8.2E-10   & 3.0E-09   & 3.5E-04   & 2.7E-03   & 1.1E-01   & 4.6E-01   \\\nopagebreak
%     &      & (6.9E-10) & (2.2E-09) & (1.0E-04) & (5.1E-04) & (3.9E-02) & (6.6E-01) \\
%     f11 & 500  & 9.9E-08   & 2.0E-04   & 1.9E-05   & 7.8E-04  & 2.0E-01   & 1.3E-02   \\\nopagebreak
%     &      & (6.0E-07) & (1.4E-03) & (5.8E-05) & (1.2E-03)  & (1.1E-01) & (1.7E-02) \\
%     f12 & 500  & 4.6E-17   & 3.8E-16   & 1.6E-07   & 1.9E-05   & 1.2E-02   & 1.9E-01   \\\nopagebreak
%     &      & (1.9E-16) & (8.3E-16) & (1.5E-07) & (9.2E-06) & (1.0E-02) & (3.9E-01) \\
%     f13 & 500  & 2.0E-16   & 1.2E-15   & 1.5E-06   & 6.1E-05   & 7.5E-02   & 2.9E-03   \\\nopagebreak
%     &      & (6.5E-16) & (2.8E-15) & (9.8E-07) & (2.0E-05) & (3.8E-02) & (4.8E-03) \\
%     \bottomrule %%% нижняя линейка
% \end{longtabu} \endgroup

% \section{Стандартные префиксы ссылок}\label{app:B4}

% Общепринятым является следующий формат ссылок: \texttt{<prefix>:<label>}.
% Например, \verb+\label{fig:knuth}+; \verb+\ref{tab:test1}+; \verb+label={lst:external1}+.
% В~таблице \cref{tab:tab_pref} приведены стандартные префиксы для различных
% типов ссылок.

% \begin{table}[htbp]
%     \captionsetup{justification=centering}
%     \centering{
%         \caption{\label{tab:tab_pref}Стандартные префиксы ссылок}
%         \begin{tabular}{ll}
%             \toprule
%             \textbf{Префикс} & \textbf{Описание} \\
%             \midrule
%             ch:              & Глава             \\
%             sec:             & Секция            \\
%             subsec:          & Подсекция         \\
%             fig:             & Рисунок           \\
%             tab:             & Таблица           \\
%             eq:              & Уравнение         \\
%             lst:             & Листинг программы \\
%             itm:             & Элемент списка    \\
%             alg:             & Алгоритм          \\
%             app:             & Секция приложения \\
%             \bottomrule
%         \end{tabular}
%     }
% \end{table}


% Для упорядочивания ссылок можно использовать разделительные символы.
% Например, \verb+\label{fig:scheemes/my_scheeme}+ или \\ \verb+\label{lst:dts/linked_list}+.

% \section{Очередной подраздел приложения}\label{app:B5}

% Нужно больше подразделов приложения!

% \section{И ещё один подраздел приложения}\label{app:B6}

% Нужно больше подразделов приложения!

% \clearpage
% \refstepcounter{chapter}
% \addcontentsline{toc}{appendix}{\protect\chapternumberline{\thechapter}Чертёж детали}

% \includepdf[pages=-]{Dissertation/images/drawing.pdf}
        % Приложения

% \setcounter{totalappendix}{\value{chapter}} % Подсчёт количества приложений

\end{document}
